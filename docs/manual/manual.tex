\documentclass[12pt]{article}
\usepackage{amsmath,array,longtable}

\newcommand{\xx}{{\mathbf{x}}}
\newcommand{\vv}{{\mathbf{v}}}
\renewcommand{\SS}{{\mathcal{S}}}
\newcommand{\BB}{{\mathcal{B}}}
\newcommand{\TT}{{\mathcal{T}}}

\title{Manual for Ves2D}
\author{Bryan Quaife}

\begin{document}

\maketitle

%%%%%%%%%%%%%%%%%%%%%%%%%%%%%%%%%%%%%%%%%%%%%%%%%%%%%%%%%%%%%%%%%%%%%%%%
\section{Brief Overview}
Given a vesicle $\xx$ with tension $\sigma$, the code simulates the
dynamics governed by
\begin{align*}
  \frac{d\xx}{dt} &= \vv_{\infty}(\xx)+\kappa_{b}\SS\BB\xx+\SS\TT\sigma, \\
  0 &= \xx_{s} \cdot \frac{d\xx_{s}}{dt}.
\end{align*}
$\vv_{\infty}$ is the background velocity which can either be a far
field condition or come from a double-layer potential due to sources on
a solid wall.  If using the latter, there is an additional second-kind
integral equation as there is an unknown density function on the
boundary of the solid walls.

%%%%%%%%%%%%%%%%%%%%%%%%%%%%%%%%%%%%%%%%%%%%%%%%%%%%%%%%%%%%%%%%%%%%%%%%
\section{Options}

\begin{longtable}{|l|l|m{6cm}|}
\hline
Name & Default & Description \\ \hline
%%%  
order & 1 & 
Time Stepping Order.  Can be 1 up to 4 \\  \hline
%%%
expectedOrder & 2 & 
If using time adaptivity, this is the order that we expect out of the
time integrator.  This plays a role in selecting the new time step size
\\ \hline
%%%
inextens & `method2' &  
Method for discretizing the inextensibility condition.  `method1'
discretizes $\mathrm{div}_{\gamma} \mathbf{v} = 0$ where $\mathbf{v}$ is
the given velocity field and `method2' discretizes
$\mathrm{div}_{\gamma} \xx^{\mathrm{new}} = \mathrm{div}_{\gamma}
\xx^{\mathrm{old}}$ \\ \hline
%%%
farField & `shear' &
Unbounded boundary condition or the name of the confined flow.
Examples include `extensional', 'taylorGreen', and 'couette'.  See
\emph{tstep.bgFlow} for all flows \\ \hline
%%%
farFieldSpeed & 1 & 
Scales the background velocity from its default value \\ \hline
%%%
vesves & `implicit' &  
Vesicle-vesicle and vesicle-boundary interactions are treated explicitly
or implicitly \\ \hline
%%%
near & true & 
Near-singular integration algorithm to handle suspensions where vesicles
are close to one another or to solid walls \\ \hline
%%%
nearStrat & `interp' &
Method for handing the near-singular integration.  Gleb (Courant) had
a first implementation of the method `cauchy' which uses the work of
Barnett and Veerapaneni \\ \hline
%%%
fmm & false &
Use the fast-multipole method to evaluate Stokes single-layer potential,
the layer potentials for the pressure, and the laplace double-layer
potential used for collision detection \\ \hline
%%%
fmmDLP & false &
Use the fast-multipole method to evaluate Stokes double-layer potential
\\ \hline
%%%
confined & false &
Determines if the flow is confined or unbounded \\ \hline
%%%
usePlot & true &
Plots the suspension as the simulation runs \\ \hline
%%%
track & false &  
Places tracker points on the vesicles for visualization.  Particularly
useful for observing tank-treading \\ \hline
%%%
quiver & false &  
Do a quiver plot of the velocity field on top of the vesicles \\ \hline
%%%
axis & [-5 5 -5 5] &  
Axis for the plot \\ \hline
%%%
saveData & true &  
Save information to a log file and the configuration to a dat file so
that it can be post-processed and visualized.  Also saves the pressure
and stress if they are being computed \\ \hline
%%%
logFile & `output/example.log' &  
Name of log file where information is saved.  This also controls the
name of the pressure/stress binary files and the profile directory \\ \hline
%%%
dataFile & `output/exampleData.bin' &  
Name of bin file where configuration is saved \\ \hline
%%%
verbose & true &  
Used to turn on or off the information being printed on the console.
Output is identical to what is written to the log file \\ \hline
%%%
profile & false &  
Run matlab's profile on the simulation and save the output to a
collection of html files \\ \hline
%%%
collision & false &  
Use Laplace double-layer potential to detect collisions.  This will
only issue a warning if vesicle cross \\ \hline
%%%
timeAdap & false &  
Use time adaptivity \\ \hline
%%%
bending & false &
Make the modification to the bending where the curve tries to fit itself
to a prescribed configuration \\ \hline
%%%
tracers & false &
Passive tracers included in the simulation \\ \hline
%%%
correct & false &
A correction to the right-hand side when doing time stepping if doing an
SDC correction \\ \hline
%%%
pressure & false &
Compute the pressure at a user-defined set of points \\ \hline
%%%
orderGL & 2 &
Number of Gauss-Lobatto points if monitoring or using the residual to
use SDC \\ \hline
%%%
nsdc & false &
Compute the residual at each time step using Gauss-Lobatto points \\  \hline
%%%
adhesion & false & 
Add in an adhesive force which is treated fully explicitly \\ \hline
%%%
periodic & false & 
If points ``leave" the geometry on the right side, reinsert them on the
left to maintain a uniform volume fraction \\ \hline
%%%
correctShape & false & 
At each time step, locally correct the vesicle's area and length to
maintain stability for long time horizons \\ \hline
%%%
antiAlias & false & 
Use anti-aliasing when computing layer potentials and traction jumps.
This is not yet fully implemented.
\\ \hline
%%%
redistributeArcLength & false & 
If the Jacobian of the vesicle shape deviates too far from a constant
value, redistribute the points equispaced in arclength.  This is an
$\mathcal{O}(N^{2})$ operation \\ \hline
%%%
\caption{Options used in Ves2D}
\end{longtable}



%%%%%%%%%%%%%%%%%%%%%%%%%%%%%%%%%%%%%%%%%%%%%%%%%%%%%%%%%%%%%%%%%%%%%%%%
\section{Parameters}

\begin{longtable}{|l|l|m{8cm}|}
\hline
Name        & Default                  & 
  Description \\ \hline
%%%
N & 64 &  
Number of points per vesicle.  Must be the same for all vesicles \\ \hline
%%%
nv & 1 &  
Number of vesicles.  This will be overwritten if user requests a
specified volume fraction when defining initial configuration \\ \hline
%%%
T & 1 &  
Time horizon \\ \hline
%%%
m & 100 &  
Number of time steps.  This only controls the initial time step size if
using adaptive time stepping \\ \hline
%%%
Nbd & 0 &  
Number of points per solid wall.  This must be the same for all solid
walls \\ \hline
%%%
Nbdcoarse & 0 &  
Number of points per solid wall on a coarse grid.  This could be used
with multigrid to precondition solid wall calculations, but it is not
fully implemented \\ \hline
%%%
nvbd & 0 &  
Number of solid wall components \\ \hline
%%%
kappa & $1e-1$ &  
Bending coefficient.  This only sets the time step size.  Must be the
same for all vesicles \\ \hline
%%%
viscCont & 1 & 
Ratio of the viscosity inside the vesicle to outside the vesicle.  May
be different for each vesicle \\ \hline
%%%
gmresTol & $1e-12$ &  
GMRES tolerance for solving all linear systems \\ \hline
%%%
gmresMaxIter & 50 &
Maximum allowable number of GMRES iterations per call to gmres \\ \hline
%%%
errorTol & $1e-1$ &  
Tolerance that the error in area and length are allowable to achieve
before the simulation stops \\ \hline
%%%
rtolArea & $1e-2$ &  
Tolerance for the local change in the error in area for adaptive time
stepping \\ \hline
%%%
rtolLength & $1e-2$ &  
Tolerance for the local change in the error in length for adaptive time
stepping \\ \hline
%%%
betaUp & 1.5 &
Maximum amount the time step size is allowed to be upscaled \\ \hline
%%%
betaDown & 0.6 &
Maximum amount the time step size is allowed to be downscaled \\ \hline
%%%
alpha & sqrt(0.9) &
Amount the optimal time step size is multiplied by to form the actual
time step size taken \\ \hline
%%%
adStrength & 4e-1 & 
Strength of the adhesive potential \\ \hline
%%%
adRange & 8e-1 & 
Width of the adhesive potential \\ \hline
%%%
\caption{Parameters used in Ves2D}
\end{longtable}
%table with name, default value and description

%%%%%%%%%%%%%%%%%%%%%%%%%%%%%%%%%%%%%%%%%%%%%%%%%%%%%%%%%%%%%%%%%%%%%%%%
\section{Classes}
\paragraph{\emph{capsules:}}
\begin{itemize}
\item Generic class for storing a vesicles, solid walls, tracers, or
other arbitrary target locations (ex. places to evaluate pressure and
stress)
\item Allowed pass any set of points (such as solid walls), or arbitrary
points (such as tracers).  However, if the curves are not closed, the
derivatives will be useless as they are computed with FFTs
\item Computes the traction jump 
\item Computes the surface divergence of of a function defined on a
curve
\item Computes the ahesion force
\item Computes the modified bending term that comes from spontaneous
curvature
\item Computes bending, tension, and divergence in matrix form so that
we can use block-diagonal preconditioner
\item Given a vesicle shape and the solid walls, computeSigAndEta
computes the tension and density function on the solid walls.  This is
required for the first time step when doing sdc corrections and when
computing the velocity field at arbitrary points (ex. Gokberk's
evaluation of the velocity on a Cheyshev-Fourier grid.
\item Near-singular integration structure is constructed.  This needs to
be improved as there is a loop over target points
\item Does collision detection
\item Can sort a collection of arbitrary points as being inside or
outside of a configuration
\item Computes stress and pressure of a given configuration.  Stress of
double-layer potential may have a bug
\end{itemize}

\paragraph{\emph{curve:}}
\begin{itemize}
\item Does basic calculus on 2-dimension curves such as finding the curvature,
jacobian, tangent vector, etc.
\item Finds the area, length, and reduced area of closed curves
\item Structure of curves is a column vector with the x-coordinate
followed by the y-coordinate.  Additional rows correspond to different
curves
\item Curves need not be closed, but then the derivatives don't make
sense 
\item Has precomputed initial shapes and solid wall configurations
\item addAndRemove is used if periodic==true to keep the volume fraction
constant
\item correctAreaAndLength finds a nearby vesicle configuration that has
the correct area and length
\item redistributeParameterize redistributes the shape, tension, and
velocity so that it is equispaced in arclength
\item Has standard spectral and local restriction and prolongation
\end{itemize}

\paragraph{\emph{fft1:}}
\begin{itemize}
\item Does FFT calculations on closed curves
\item Functions are Fourier interpolation and Fourier differentiation
\item Computes Fourier differentiation matrix
\end{itemize}

\paragraph{\emph{monitor:}}
\begin{itemize}
\item Handles all the input and output of the results
\item Initializes files when starting a new run
\item Checks the error in area and length and terminates the code if it
is above a certain tolerance
\item Writes messages to log files and console
\item Writes vesicle information to dat file
\item Writes the pressure and stress to dat files
\item Plots the vesicle configuration
\item Gives a final summary of the run
\end{itemize}

\paragraph{\emph{poten:}}
\begin{itemize}
\item Does standard potential theory for closed curves
\item Loads and uses quadrature rules for single- and double-layer
potentials for stokes and laplace
\item Contains main near-singular integration function
\item Computes single- and double-layer potentials for Stokes and
Laplace.  When available, there are routines that use the FMM.
\item Many layer potentials can be generated as matrices to do
self-evaluation
\item Computes layer potentials for pressure and stress
\item Gleb's implementation of the near-singular strategy of Shravan and
Barnett is in this routine.  This is not yet operational
\end{itemize}

\paragraph{\emph{tstep:}}
\begin{itemize}
\item Class to move the simulation forward in time
\item Has different time integrators such as first- and second-order,
implicit vs. explicit vesicle-vesicle interactions, two different
methods for enforcing the inextensibility.  Note that explicit is not
fully supported.  It was not very robust, so it has not been maintained
\item Includes a small time step size initialization for doing
second-order time stepping
\item Can take Gauss-Lobatto sub time steps for computing the residual,
doing adaptive time stepping, or simply to study a different time
integrator
\item Computes the different velocities due to the different sources on
all sources.  For example, vesicle to vesicle, vesicle to wall, stokeslets to
vesicles, etc.
\item Applies the block-diagonal preconditioner
\item Computes the velocity at tracers
\item Has all the background unbounded velocities
\item Computes the residual at Gauss-Lobatto points which can either be
simply monitored or used to build an adaptive time stepping scheme
\item Does Gauss-Lobatto integration
\end{itemize}

\end{document}

