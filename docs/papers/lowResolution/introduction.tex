Main issues that need to be resolved are
\begin{itemize}
  \item Filtering, aliasing, and other Fourier issues for very coarse
  resolutions:  Suggestion: use $2\times$ upsampling for product of
  functions and $n\times$ upsampling for any nonlinear function. Since
  the cost becomes $n^2$ it is  still much less than $n^3$ required
  for the block diagonal preconditioner. If it is too expensive we can
  reduce it to $n/2$ or $n/4$ upsampling. 
  \comment{Bryan: I've seen that upsampling by a factor of 2 (or maybe
  even 1.5) with Fourier restriction is adequate.  This is used to
  compute the layer potentials as well as the bending, tension, and
  surface divergence.  I only upsample the vesicle position and the
  function of interest (tension, density, etc).  Then, terms like the
  jacobian are computed using the upsampled vesicle positions.  This is
  in contrast to upsampling the jacobian from the lower resolution; this
  expressions has aliasing errors in it, too.}

  \item Equidistribution in arclength: this works well.
  \comment{Bryan: This does work well, and it is key to equally
  distribute the tracker points of the vesicle in arclength.  However, I
  have examples where the points start to cluster and this leads to an
  instability.  Therefore, I now have an option to check the jacobian at
  each time step, if it deviates from a constant value by more than some
  threshold, then the vesicle is reparameterized so that the points are
  equally distributed in arclength.  This is a $\mathcal{O}(N^{2})$
  algorithm, but I expect that this'll beat one possible alternative
  which would require a non-uniform FFT.}

  \item Local correction to area and length of vesicle at each time
  step:  for this we have to do high-fidelity simulations to show that
  we should always use it.
  \comment{Bryan: It's easy to show that it is essential to correct the
  vesicle's area and length.  Even in dilute concentrations (1 vesicle),
  if there is a long time horizon, this is essential.}

  \item Near-singular integration at very coarse resolutions: This will
  create problems and will be the most critical to resolve to avoid
  collisions.  Upsample both sources and targets to $n^{3/2}$.  Then
  perhaps try a more sophisticated filtering in which perhaps and spline
  projection to minimize Gibbs phenomena, and perhaps some filtering,
  WENO. No dynamic simulations, just look at different two-vesicle
  configurations for the near interaction.  Filtering/smoothing (not
  thresholding) may be necessary, although will more rigid.
  \comment{Bryan: I haven't observed much difficulty here.  As long as
  the layer potentials are evaluated with upsampling along the vesicle
  boundary and at the interpolation nodes, I think this'll be fine.  One
  thing that I plan to do is only use 3 interpolation points, one on the
  source vesicle and two after the target points.}

  \item Should we filter overall the velocity before we add it to the
  new shape? See above; perhaps using couple filtering using
  optimization with distance contraints?
  \comment{Bryan: I don't think this is necessary.  By redistributing the
  points equally in arclength, this will help control the high
  frequencies in the Jacobian, and therefore, in all the required
  derivatives, and the next time step.}

  \item Low accurate FMM so that far field is less accurate, so that FMM
  is fast. Maybe 3 digits or less. 
  \comment{Bryan: This still needs to be done.}

  \item What should the GMRES tolerance be? I guess not too small since
  we want as large a step as possible because we want to reuse
  preconditioner. Coupling it with several steps of SDC may help to
  amortize cost. Perhaps even increase number of SDC sweeps to go
  towards a fully implicit scheme that may improve perhaps stability.
  May be will improve acceptance rates (assuming second-order time
  stepping).
  \comment{Bryan: The GMRES tolerance is tricky.  The block-diagonal
  preconditioner means that GMRES mostly resolves inter-vesicle
  interactions.  If we are using a low-accurate FMM, then it may be
  appropriate to use a low GMRES tolerance.  As per SDC, what seems to
  work best so far is to use only 2 Gauss-Lobatto points (ie. no
  substeps), and then do SDC sweeps to get a more accurate first-order
  solution.  Results for this in the single vesicle shear run are soon
  to be reported.}

  \item Changing the bending coefficient doesn't make a different for a
  single vesicle, it just changes the time scale and thus the number of
  time steps. But what  if we have multiple vesicles?  Perhaps it makes
  a difference, especially in the presence of solid walls. 
  \comment{Bryan: It seems to me that the bending coefficient only
  matters if each vesicle has a different bending coefficient (which the
  code doesn't support right now).  Even if there is a solid wall, the
  time scale can be controlled by changing the magnitude of the boundary
  conditions on the solid wall}

  \item What if vesicles cross? We have to be able to detect crossing.
  If they did, move back and filter more aggressively or moving
  positions to increase distances? How should we do that? One idea is to
  increase resolution in space for a few iterations. But if we have
  concentrated suspensions, it may happen every a few time steps.
  Another alternative is to increase repulsion?
  \comment{Bryan: We'll have to wait and see what happens when I have
  multiple vesicle simulations.}

  \item Adding repulsion. What should be the form of repulsion and what
  should be its length scale. If we're upsampling for the near singular
  evaluation. Can we adjust the repulsion length-scale adaptively,
  depending on the distance of two vesicles, so you don't have a
  constant force. 
  \comment{Bryan: There is also the issue of the magnitude of the
  repulsion.  For adhesion, I implemented the adhesive force between a target
  point and a vesicle as an average value (integral) of the potential
  over all source points on the vesicle.  So, for repulsion, as two
  vesicles come closer together, the repulsion will naturally increase.
  I am assuming this is what is meant by adjusting the length-scale
  adaptively.}

  \item The preconditioner can perhaps be inverted on a coarse grid.
  Need to test what the effect on the number of GMRES iterations is.
   

\end{itemize}

What to we evaluate the coarse grid solver
\begin{itemize}
  \item Global measures: effective viscosity, normal stresses
  \item Single vesicle: inclination angle, phase diagrams for slipper
    cell and for tumbling, and for shear flow. 
  \item Two vesicles: time and drainage for extensional flow,
    separation distance for two vesicles in shear. 
  \item For suspensions: look at velocity statistics (a la pietro). 
  \item For walls/microfluidics?
\end{itemize}

Algorithms that need to be implemented in some shape or form include:
\begin{itemize}
  \item Upsampling, and maybe filtering, at all instances of
  multiplication, division, and exponentiation of periodic functions.
  \item Local fix to area and length (done)
  \item Near-singular integration at coarse grids
  \item Techniques to measure statistics of vesicle flows
  \item Look at average pressures and stresses
\end{itemize}


