Vesicles are deformable capsules filled with a viscous fluid. The
hydrodynamics of vesicles that are suspended in a viscous fluid
(henceforth, ``{\em vesicle flows}'') play an important role in many
biological phenomena~\cite{kraus1996,seifert}.  For example, they are
used experimentally to understand properties of
biomembranes~\cite{sackmann1996} and red blood
cells~\cite{noguchi2005,pozrikidis1990,ghigliotti-biros-e10,kaoui-tahiri-biros-misbah-e11,misbah2006}.
Here we discuss numerical algorithms for simulating the motion of
vesicles in Stokesian fluids in two dimensions.  Many features of
algorithms designed for particulate flows of vesicle suspensions find
applications to other deformable particulate flows, for example
deformable capsules, bubbles, drops, and elastic filaments.  Vesicle
flows are characterized by large deformations, the local
inextensibility of a vesicle's membrane, the conservation of enclosed
area due to the incompressibility of the fluid inside the vesicle, and
the stiffness related to tension and bending forces.  Efficient and
accurate numerical methods need to address these issues.  It is known
that rheology can be sensitive to elastic instabilities that need to be
resolved quite accurately in long-horizon
simulations~\cite{kaoui-biros-misbah09}. 

In~\cite{pozrikidis2001a}, Pozrikidis presents integral equation
formulations for different types of particulate and interfacial viscous
flows.  In line with our work, integral equation methods have been
applied to vesicle suspensions and other similar capsules in two- and
three-dimensions by many
groups~\cite{veerapaneni-e11,rah:vee:bir,shravan,fre:ore2011,fre:zha2010,zha:isf:ols:fre2010,zha:sha2011a,zha:sha2013a,zha:sha2013b,zha:sha:nar2012},
to name a few.  We focus on algorithms for vesicle flows from our
group's previous work~\cite{shravan}, where numerical algorithms for a
boundary integral formulation of the vesicle hydrodynamics are
presented.  In this formulation, a Stokes single-layer potential is
coupled with the vesicle membrane forces and results in an
integro-differential equation that is constrained by the local
inextensibility constraint (see Section~\ref{s:formulation} for
details).  In~\cite{rah:vee:bir}, these methods were extended to
confined flows and problems with vesicles whose enclosed fluid has a
viscosity contrast with the bulk fluid.  In~\cite{veerapaneni-e11}, we
developed algorithms for three-dimensional vesicle flows.  To summarize,
the main components of the formulation we have used in our prior work
are the following: (1) We proposed a time-stepping method in which the
inter-vesicle interactions are treated explicitly and the self-vesicle
interactions are treated semi-implicitly to remove the bending and
tension stiffness; (2) We proposed a spectrally accurate discretization
in space for differentiation and integration of smooth functions and
weakly singular integrals; (3) We proposed effective preconditioners for
the self-vesicle interactions; and (4) We used the fast multipole method
to accelerate the integral operators that capture the long range
hydrodynamic interactions.


\paragraph{Contributions}
In this paper, we continue our efforts towards efficient algorithms for
the simulation of vesicle hydrodynamics using boundary integral
formulations.  We focus on specific issues that are common when
simulating high concentration vesicle flows.  In particular, we look
into stiffness due to {\em inter-vesicle} dynamics, spectrally accurate
collision detection, near-singular integration, and calculating
quantities like pressure and stress fields that can be used to define
coarse-grained variables.  To make things somewhat more precise, let
$\xx_{p}$ be the shape parametrization of the $p^{\mathit th}$ vesicle
(e.g., Lagrangian points or Fourier coefficients), and let $\uu_{p}$ be
its velocity. Then the evolution equation for the position of the
boundaries of $M$ vesicles can be summarized by
\begin{align}
 \frac{d \xx_p}{d t} =\uu_\infty(\xx_p) + \sum_{q=1}^M \uu_{p}(\xx_{q}), \quad
p=1,\ldots,M,
\label{e:highlevel}
\end{align}
where $\uu_{p}(\xx_{q})$ is the velocity induced on vesicle $p$ due to
forces on vesicle $q$ (via hydrodynamic interactions) and
$\uu_\infty(\xx_p)$ is an external velocity field that drives the flow.
The term $\uu_{p}(\xx_{p})$ is the self-interaction term for the vesicle
$p$ and $\uu_{p}(\xx_{q})$, $q \neq p$, are the inter-vesicle
interactions.

\begin{itemize}
\item {\bf Implicit inter-vesicle interactions.} Most research groups
working on vesicle simulations discretize~\eqref{e:highlevel} using an
explicit time-stepping method. However,~\eqref{e:highlevel} is quite
stiff due to tension and bending.  For this reason, in~\cite{shravan}
we introduced time-marching schemes that treat the self-interactions
semi-implicitly.  For example, a first-order accurate in time scheme
(with $\uu_\infty=0$) reads
\begin{align*}
  \frac{\xx_p^{t+dt}-\xx_p^t}{dt} = \tilde{\uu}_p(\xx_p^{t+dt}) + 
    \sum_{\substack{q=1 \\ q \neq p}}^{M} \uu_{p}(\xx_{q}^{t}), 
      \quad p=1,\ldots,M. 
\end{align*}
The velocity $\tilde{\uu}_{p}(\xx_{p}^{t+dt})$ indicates a linearized
semi-implicit approximation of the vesicle self-interaction term
$\uu_{p}(\xx_{p})$.  Thus, computing $\xx_{p}^{t+dt}$ costs one linear
solve per vesicle.  This is sufficient to reduced bending and tension
stiffness but can be problematic for concentrated suspensions since
$\uu_{p}(\xx_{q})$ also introduces stiffness when vesicles $p$ and $q$
are near each other.  In this paper, we introduce a semi-implicit
scheme of the form
\begin{align*}
  \frac{\xx_p^{t+dt}-\xx_p^t}{dt} = \sum_{q=1}^M 
    \tilde{\uu}_p(\xx^{t+dt}_{q}), \quad p=1,\ldots,M,
\end{align*}
that treats all coupling terms in a linear semi-implicit way.  Thus,
one needs to solve a coupled system for all vesicles at every time
iteration.  We examine the stability of this scheme for first- and
second-order time-stepping and we compare it with alternative
time-marching methods.

\item {\bf Near-singular integration.} When the vesicles come closer
together, the hydrodynamic interaction terms (e.g., $\uu_{p}(\xx_{q})$)
require the evaluation of near-singular integrals.  This can be quite
expensive computationally.  Here we use the two-dimensional analogue of
the scheme introduced in~\cite{ying-biros-zorin06} which allows a
fifth-order accurate evaluation of near interactions using optimal work.

\item {\bf Spectral collision detection.} Another issue related to
concentrated suspensions is collision detection.  While it is
physically impossible for vesicles to collide with one another or with
solid walls in finite time, it is important to detect collisions caused
by numerical errors.  There exist many efficient algorithms for
computing intersections for polygonal domains~\cite{jimenez-e13}.
Since our spatial discretization is done in high-order accuracy and the
vesicle boundaries are $C^\infty$ curves that are spectrally resolved,
we propose a collision detection scheme that is based on potential
theory and it is ideally suited for smooth geometries.  The detection
can be done in linear time in the number of degrees of freedom used to
represent the vesicle boundaries and requires one evaluation of the
free-space Laplace potential.  The method is described in
Section~\ref{s:collision}.

\end{itemize}
We test these methodologies in a variety of flows and we study the
stability and accuracy of the overall algorithm.  As a secondary
contribution, we also provide a scheme for computing average pressures
and stresses in regions of interest, described in
Section~\ref{s:aver}.  The results of our tests are described in
Section~\ref{s:results} in which we test problems with viscosity
contrast, confined flows, and concentrated suspension flows.

\paragraph{Limitations} 
The main limitation is that the method is developed in two dimensions.
Although several flows can be described to good accuracy under a
two-dimensional approximation, a lot of interesting phenomena occur
only in three-dimensional, especially for concentrated suspensions.
The extensions that we present in this paper do not rely on
two-dimensions, and using these methods in three-dimensions is a
problem of implementation.  In particular, the spectral collision
detection scheme only relies on a standard potential theory result that
holds in three-dimensions, and the fast multiple method.  The time
integrators also naturally extend to three-dimensions, but more work is
required to precondition the linear operators that appear upon time
discretization.  The near-singular integration details have been
discussed for generic surfaces in~\cite{ying-biros-zorin06}, but
several optimizations are possible for shapes represented by spherical
harmonics, which we use in our three-dimensional schemes.

Another major limitation is that we do not use time and space
adaptivity.  Those two are essential for efficient robust solvers that
can be used by non-experts.  Currently, we select the time step size by
a trial and error process.  We are currently working on devising time
and space adaptive methods.

Finally, the method is most suitable for Stokesian bulk fluids.  If
inertial or viscoelastic effects are important, a boundary integral
formulation cannot be used.

\paragraph{Related work} 
There is extensive work on vesicle simulations.  Our list here is by no
means exhaustive, but it includes some of the work that is most relative
to ours.  Pozrikidis offers an excellent review of numerical methods for
interfacial dynamics in a Stokes flow~\cite{pozrikidis2001a}.  Capsules
similar to ours have been simulated in concentrated suspensions
in~\cite{li:poz2002,zha:sha:nar2012,zha:sha2011b,zha:isf:ols:fre2010,fre:zha2010,fre:ore2011}.
Also, in~\cite{pra:riv:gra2012,kum:riv:gra2014,kum:gra2011}, high
concentration suspensions of a different kind of capsule are considered,
but these capsules do not resist bending; therefore, the governing
equations are far less stiff.  While some of these methods use spectral
methods, most of them use an explicit time stepping method which results
in a strict restriction on the time step size.  In order to allow for
larger time steps, a Jacobian-free Newton method outlined
in~\cite{dim:hig1997} was applied to an implicit discretization of
droplets submerged in a Stokes flow~\cite{dim2007}.  While this allows
for larger time steps, it requires the solution of nonlinear equations
which can be computationally expensive, and this technique has not been
tested on vesicle suspensions.  Work that uses a combination of implicit
and explicit methods to study the dynamics of a single vesicle
include~\cite{zha:sha2013a,zha:sha2009}, but both these methods require
multiple solves per time step since they use a predictor-corrector
scheme.  In this paper, we remove the stiffness with a semi-implicit
method which only requires solving one linear equation per time step.
These methods have been applied to a single vesicle
in~\cite{zha:sha2009,sal:mik2012} and are extended to multiple vesicles,
where the only the self-interactions are treated semi-implicitly
in~\cite{rah:vee:bir,shravan,zha:sha2013b}.  However, with multiple
vesicles, we are unaware of any work that couples all the vesicles and
solid walls implicitly and  achieves more than first-order accuracy in
time.

Evaluating layer potentials close to their sources is an active area of
research.  Popular methods to evaluate near-singular integrals include
upsampling or high-order quadrature
rules~\cite{helsing-ojala08,kro1999}, and singularity subtraction or
partitions of unity~\cite{poz1999,fre:zha2010,zha:isf:ols:fre2010}.
Some of these methods claim to achieve up to third-order accuracy, but
they do not report timings or accuracies.  Therefore, the cost required
to achieve the accuracies we desire is unclear.  Moreover, these methods
often depend on the dimension of the problem and nature of the
singularity.  A more recent technique is QBX~\cite{klo:bar:gre:one2012}
which can deliver arbitrary accuracy, but is difficult to implement and
we anticipate it will be too expensive for problems with moving
boundaries.  The method we use naturally extends to three dimensions and
we have successfully used it for a variety of near-singular integrals.

\paragraph{Outline of the paper} 
In Section~\ref{s:formulation}, we summarize the formulation of our
problem. In Section~\ref{s:method}, we discuss the spatio-temporal
discretization including the new scheme that semi-implicitly couples the
vesicle updates, the near-singular evaluation
(Section~\ref{s:near-singular}), and the collision detection
(Section~\ref{s:collision}) .  The average pressure and stress
calculations are in Section~\ref{s:aver} and the results are described
in Section~\ref{s:results}.



