\documentclass[11pt]{article}


\usepackage{fullpage}
\usepackage{amsmath,amsfonts,amssymb,stmaryrd}
\usepackage{color}
\newcommand{\comment}[1]{{\color{blue} #1}}
\newcommand{\todo}[1]{{\color{red} #1}}
\newcommand{\note}[1]{{\color{green} #1}}
\newcommand{\grad}{{\triangledown}}
\newcommand{\nn}{{\mathbf{n}}}
\newcommand{\uu}{{\mathbf{u}}}
\newcommand{\xx}{{\mathbf{x}}}
\newcommand{\DD}{{\mathcal{D}}}
\renewcommand{\SS}{{\mathcal{S}}}

\begin{document}

\section*{Reviewer 1}

\comment{The authors cite their 2010 paper for much of the background
material.  This has several problems:}
\begin{itemize}
  \item While we understand the reviewers concern, the majority of our
  current work focuses on algorithms that extend our 2010 work to
  concentrated suspensions.  It was not our intention to cite our 2010
  paper for the background material.  We did state that the literature
  had been reviewed in our previous work, and we referred the readers
  to these papers for a formal literature review.

  \item We have decided to do a much more complete literature review
  in this work that cites work listed in our 2010 paper and other
  older and more recent work.
\end{itemize}

\comment{
There are extensive simulations of high-volume fraction
configurations in two dimensions (Pozrikidis, Freund, Higdon), some of
which have use spectral methods. The authors even go so far as to
presume that their methods ``will enable simulations with concentrated
suspensions.'' This blatantly denies the studies that others have
already done this.}
\begin{itemize}
  \item Several citations that involve concentrated suspensions are
  now mentioned. The reviewer was correct to point out that previous
  work has been done. Our goal is to enable larger time steps and
  hopefully faster algorithms. 
\end{itemize}

\comment{This fails to appreciate that there have also been extensive
successful simulations of vesicle suspensions even in three dimensions
(Shaqfeh, Graham, Freund).}
\begin{itemize}
  \item Several citations that discuss vesicle suspensions, both in
  the concentrated and dilute limit, and in two- and three-dimensions,
  are now included. Actually, our group was the first to develop
  semi-implicit algorithms for vesicles in three dimensions, reference
  [41].  Freund has not worked on vesicles before (deformable capsules
  are different from vesicles because they have shear resistance). At
  any rate, our focus here is the investigation of stiffness and the
  need for fully implicit methods. We clarify this in the paper.
\end{itemize}

\comment{With the citations listed, the authors seem to credit themselves
for development of the Stokes-flow boundary integral formulation and
fundamental phenomenology such as tank treading of the vesicle
membrane.}

That was not our intention. In our earlier work, reference [27], we
give a detailed overview of the prior work in this field. Our
introduction has been modified to clarify this:
\begin{itemize}
  \item For tank treading, we now cite {\em V. Kantsler and V.
  Steinberg ``Transition to tumbling and two regimes of tumbling motion
  of a vesicle in shear flow'', Physical Review Letters, 96(3), 2006}
  and {\em C. Misbah ``Vacillating breathing and tumbling of vesicles
  under shear flow'', Physical Review Letters, 96 (2), 2006}.  These
  references were used in our group's earlier work where we simulated
  tank treading and tumbling.

  \item Also we cite {\em C. Pozrikidis ``Interfacial dynamics for
  Stokes flow'' Journal of Computational Physics'', 169:250-–301,
  2001} in the second paragraph of the paper and also include a list
  of references that have applied integral equations to vesicle
  suspensions.
\end{itemize}

\comment{Others have used near-singular formulations for vesicle
simulations (Higdon, Zhao).}

\begin{itemize}
  \item These works are very recent, are third-order schemes, and we
  expect that they would be computationally prohibitive if a
  higher-order scheme were used. There exist other sophisticated methods
  developed by our and other groups, but either they are expensive or
  they do not generalize in 3D.  (In 2D the problem is solved by
  Helsing, reference [9]. Unfortunately it doesn't generalize to 3D.)
  We have included a list of references that discuss high-order
  quadrature, singularity subtraction, partitions of unity, and
  quadrature by expansion (QBX).  Some of these references address
  near-singular integrals in the context of interfacial dynamics.
  \end{itemize}

\comment{Others have developed time-implicit methods
(Zhao,Dimitrakopoulos)}
\begin{itemize}
  \item Dimitrakopoulos did not study vesicles but droplets.  The
  governing equations are different and we are unaware of an work that
  uses a Jacobian-free Newton method to simulate vesicle suspensions.

  \item Zhao has developed implicit schemes for a single vesicle in
  {\em Hong Zhao and Eric S.~G.~Shaqfeh ``The shape stability of a
  lipid vesicle in a uniaxial extensional flow'', Journal of Fluid
  Mechanics, 719, 2013} and {\em H.~Zhao and E.~S.~G.~Shaqfeh ''The
  dynamics of a vesicle in shear flow" 2009}.  Multiple vesicles were
  considered in {\em Hong Zhao and Eric S.~G.Shaqfeh ``The dynamics of
  a non-dilute vesicle suspension in a simple shear flow" Journal of
  Fluid Mechanics, 725, 2013}, but their time integrator is
  first-order, treats inter-vesicle interactions explicitly, and
  requires two solves since they use a predictor-corrector scheme.

  \item Indeed, we are not aware of any other implicit schemes for
  vesicle flows that treat inter-vesicle interactions implicitly.  We
  have included a list of references that discuss fully-implicit and
  semi-implicit methods. Our goal is to study the stability of the
  schemes.
  \end{itemize}

\comment{Given that so many others are already doing effective
simulations in three dimensions, this extension needs to be addressed
more completely. Extension to three-dimensions should be discussed more
explicitly after the present formulation is presented.}
\begin{itemize}
  \item We have mentioned at several locations ({\em Limitations} and
  {\em Conclusions}) that  our scheme does not depend on
  the number of spatial dimensions.  In particular, near-singular
  integration, fast spectral collision detection, and semi-implicit
  inter-vesicle interactions can all be trivially formulated in 3D.  The
  issue is one of a careful implementation that takes advantage of the
  spherical harmonics we use to represent the surfaces.

  \item The larger challenge in 3D is preconditioning the linear system
  resulting from our new time integrator.  We mention this problem and
  do not address it further in this work.
  \end{itemize}

\comment{Page 5: $f$ seems to be a linearized constitutive model. How
important is this for the implicit scheme? This must be addressed,
especially since others have done so many successful simulations
(including implicit) with significantly nonlinear finite-deformation
constitutive models.}
\begin{itemize}
  \item The constitutive law for the elasticity of the membrane is
  nonlinear in the position of the membrane, but the elastic energy is
  a simple quadratic of the sum of curvature and  tension.  We now state
  that $f$ is nonlinear in the position. Note that this is not our
  model, but a standard constitutive scheme that has been used for
  vesicles for many years and has been experimentally validated.

  \item The main challenge with fully implicit is the global coupling
  with the fluid and making sure the algorithm is fast enough.  We have
  mentioned this in the text.
\end{itemize}

\comment{Page 6: in 2d, others have removed the singularity and treated
it analytically. Why isn’t this done here? It would seem to then provide
spectral accuracy. The remaining sharpish though $C^{\infty}$ integrand
could then be treated spectrally with the present method. (I do not
think, however, that this would extend to three dimensions.)}
\begin{itemize}
  \item Singularity subtraction by itself does not provide spectral
  accuracy.  A high-order quadrature rule is also required.  However,
  if adaptive quadrature is used, especially when the vesicles are
  nearly touching, the scheme may become overly expensive.  Also,
  singularity subtraction usually depends on the kernel.  We are
  interested in a near-singular integration technique that can robustly
  be applied to all they layer potentials required (single- and
  double-layer for Stokes, double-layer for Laplace, pressure and
  stress of Stokes).  In addition, our near-singular integration
  technique is fast, easy to implement, and the trapezoid rule is the
  only quadrature formula required. 

  \item We have included additional references for near-singular
  integration in the {\em Related work} section, including singularity
  subtraction.
\end{itemize}

\comment{Before (1): $\gamma_{p}$ seems to be introduced prematurely}
\begin{itemize}
  \item $\gamma_{p}$ is not defined until just before Equation (2)
  where it is first used.
\end{itemize}

\comment{It would be good to state from the outset that collision detection is
only needed to counteract numerical errors. Physical collisions should
not occur in the Stokes limit.}
\begin{itemize}
  \item This is mentioned in the introduction
\end{itemize}

\comment{Page 3. $f(\xx) = [T]\nn$ does not seem to be well formed. Is
$f$ a stress or a force, vector or rank-2 tensor, ...?}
\begin{itemize}
  \item $f$ is the sum of two forces: one due to bending and the other
  due to tension.  This has been clarified.
\end{itemize}


\comment{Equation (2): $\Delta$ is some sort of vector Laplacian? It’s
not a usual scalar Laplacian. I personally prefer $\mu\grad \cdot
(\grad \uu + (\grad \uu)^{T})$, but this is not a big deal.}
\begin{itemize}
  \item The correct statement is what the one you suggest. In fact, for
  certain boundary conditions, the two are not mathematically
  equivalent. We had used Laplacian for simplicity.  We have now changed
  it. 
\end{itemize}


\comment{Equation (2): The authors seem to use both $\sigma$ and $T$ for
stress.}
\begin{itemize}
  \item $T$ represents the stress of the fluid while $\sigma$
  represents the tension of the vesicle.  This is explained in the
  paragraph leading up to Equation (2).
\end{itemize}



\comment{Page 4: The definition of $\Gamma_{0}$ is unclear/ill-formed.}
\begin{itemize}
  \item This has been clarified.
\end{itemize}


\comment{Definition of $\SS_{pq}$: ``$-I \log \rho$'' would be more
clear.}
\begin{itemize}
  \item Both instances of this notation have been changed.
\end{itemize}


\comment{Are we sure there is no $4$ in front of the $\pi$ in the
definition of $D_{pq}$?}
\begin{itemize}
  \item We used equation (2.2) from {\em Effect of membrane bending
  stiffness on the deformation of capsules in simple shear flow,
  Pozrikidis}.  By multiplying the equation by $(1+\lambda)$, we arrive
  at Equation 3a from our paper.
\end{itemize}


\comment{Page 5: ``$\omega_{1}$ is the interior of vesicle $q$''. This
must be misrepresented.}
\begin{itemize}
  \item Fixed
\end{itemize}


\comment{Page 5: $\nu_{p}$ is repeatedly defined}
\begin{itemize}
  \item It is now only defined once.
\end{itemize}


\comment{Page 5: what is meant by ``hydrodynamic density''? Is this the
local surface traction density?}
\begin{itemize}
  \item The hydrodynamic density refers to the function that
  is used in the single-layer potential.  In our case, it is the jump in
  the stress across the vesicle which is also the traction density. It
  is a standard terminology in potential theory. 

  \item We have mentioned on page 5 that the stress jump may also be
  referred to as the traction jump.

  \item We also changed the language in Section {\em Computing Local
  Averages of Pressure and Stress}.  Previously, we use ``traction
  jump'' which is inaccurate since, in our work, the density function
  can be defined on the solid walls (for confined flows) or be the
  velocity on the vesicle (for viscosity contrasts).
\end{itemize}


\comment{Page 6: ``fully decoupled'' should be explained more explicitly.
Clearly, it is not literally true for the physical system.}
\begin{itemize}
  \item We have clarified that the vesicles and solid walls being fully
  decoupled is an artifact of the numerical method.

  \item We have also clarified that equation (5) is also an
  approximation of the governing equations (3), but the benefit of (5)
  is the additional stability.
\end{itemize}


\comment{Page 7: ``Then number of GMRES....''}
\begin{itemize}
  \item Fixed
\end{itemize}


\comment{Footnote 3: This is usually presented as a dynamic condition
based upon the lubrication limit of the flow equations. I do not think
that the streamline condition can sufficiently explain evolution in
time.  (By their argument nothing should ever contact if separated by an
incompressible continuum, but that failed to appreciate the dynamics
associated with that continuum, which must be viscous, I think.)}
\begin{itemize}
  \item For an incompressible continuum two solid surfaces cannot touch
  in finite time.  The reason that in reality objects touch is that the
  incompressibility assumption (first) and then the continuum hypothesis
  (second), break down.  Lubrication theory just gives an approximation
  for the forces between the particles.  See {\em Bachelor,
  "Introduction to Fluid Mechanics," pages 75--79}, for steady-state
  incompressible flow fields.  We also state that lubrication theory
  could be used to study the forces between nearly touching vesicles. 
\end{itemize}


\comment{Section 5. I find the use of area and length to be acceptable
in assessing performance, but the authors should more explicitly
explain that these are dominated by relatively low-order moments of the
shape and therefore very easy to compute accurately. I request that
convergence of some harder quantity be added, at least for one example,
though it will be less definitive without an exact solution. I would
really like to know how the wall-vesicle distance converges for the
stenosis.}

\begin{itemize}
  \item It is true that they are low-order moments of the shape and are
  easy to compute accuratly.  However, that does not automatically
  imply that a numerical method will easily maintain these quantities.
  In fact this is not always the case.  Most methods that use penalty
  to enforce local inextensibility must constantly correct the shape
  and length otherwise the methods diverge quickly.  We provide
  experimental evidence that maintaining high accuracy on the shape is
  a good proxy on the overall accuracy since we are not do any
  numerical corrections.  However, the reviewer is right, maintaining
  area and length does not imply convergence. The convergence of the
  overall formulation has been shown in our earlier work,  references
  [33], [40], and [41]. 

  \item We added the comment that the area and length are low-order
  moments of the shape. We have found them to be very reliable for
  estimating errors.  This is also now mentioned in the text.

  \item For the extensional example, we now report the error in the gap
  size rather than the actual gap size.  The ``true'' solution is found
  by taking $N=256$ points per vesicle and a sufficiently small time
  step that the error in area and length is at least a full order of
  magnitude smaller than the most accurate reported solution.

  \item We looked at the minimum wall to vesicle distance for the
  reported runs and a convergence rate was inconclusive.  Therefore, we
  ran additional runs with the finest spatial resolution and multiple
  time step sizes.  We do observe first-order convergence, but not
  until 16,000 time steps are taken.  Without an exact solution,
  we are not able to conclusively achieve second-order convergence.
  However, we do show that the minimum distance is resolved up to at
  least three digits.
\end{itemize}


\comment{Stenosis: what is the exact shape of the wall used. It’s also
unclear how the flow is imposed. This should be explained in detail.}
\begin{itemize}
  \item A parameterization of the solid walls along with estimates for
  the trapezoid rule applied to the double-layer potential are now
  included in the text.

  \item The boundary condition on the solid wall is plotted and
  explained in the text.

\end{itemize}





%%%%%%%%%%%%%%%%%%%%%%%%%%%%%%%%%%%%%%%%%%%%%%%%%%%%%%%%%%%%%%%%%%%%%%%%%
\section*{Reviewer 2}
\comment{In section 3.4 a spectrally accurate collision detector is
described. How does it handle close-to-touching vesicles? It seems like
it could suffer from inaccuracy due to near singular integrands in such
cases (for example for high concentration examples such as the one in
Fig. 10). Is there a specific reason why you don't use your scheme for
near-singular integration here? Using an adaptive time-stepper would
likely remove this problem, so in the long run it is perhaps not a big
deal, but it could perhaps be beneficial for the reader to add some
comments about this}
\begin{itemize}
  \item The reviewer is correct that the collision detector suffers
  from the same inaccuracy due to near-singular integrands.  In fact,
  we do use near-singular integration, but this was not clear in the
  original submission.

  \item We have clarified that the collision detection uses
  near-singular integration, and, therefore, placed the collision
  detection section after the near-singular integration section
\end{itemize}

\comment{The first-order and second-order plots in the shear flow
examples are very similar. Perhaps one of the figures could be used to
to display additional data? For example difference between first and
second order distances over time, or area and length errors.}
\begin{itemize}
  \item We agree that no additional information was being provided by
  the two figures.  After looking at the errors in area and length, and
  looking at the difference between the first- and second-order
  distances, we felt that between the figures captions and Tables 4 and
  5, nothing new could be offered by a new plot.  Therefore, we have
  merged Figures 3 and 4 and the redundancy is removed.
\end{itemize}

\comment{How does the length and area errors vary with time in the
stenosis example? For $t = 5$ the vesicle has highly varying curvature,
do the errors increase at this stage or are they roughly constant? A
figure isn't necessary, but some comments would be appreciated.}
\begin{itemize}
  \item In fact, the errors jump closer to $t=3$ when the vesicle is
  more or less centered in the constriction.  We have added a paragraph
  to describe the nature of the error.
\end{itemize}

\comment{The reference blue circle in Fig. 10 seems to be missing.}
\begin{itemize}
  \item This comment is removed.  The boundary condition on the solid
  walls is described in the caption. 
\end{itemize}

\comment{Some minor textual mistakes (search for N\"{y}strom,  integralr
and ``of the its area'' in your tex file)}
\begin{itemize}
  \item Fixed

  \item We have done several additional reads for grammar and textual
  errors.
\end{itemize}

\end{document}
