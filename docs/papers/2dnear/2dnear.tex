\documentclass[preprint, 10pt]{elsarticle}

%% labeling 
%%   equations \label{e:equation_name}
%%   sections  \label{s:section_name}
%%   tables    \label{t:table_name}
%%   figures   \label{f:figure_name}
%%   alagorithms \label{a:algorithm_name}
%%   Apendix \label{A:appendix_name}

%% bibtex names: all lower case
%%   one author: biros89, ghattas91, ghattas91a, ghattas91b, ghattas91c 
%%   up to three authors:  biros-ghattas-smith99
%%   more than three authors:  biros-e98
%%
%% for graphs and figures:
%%    tikz (best) or  one of the following: matlab, xfig, ggplot,
%%     see: http://www.texample.net/tikz/examples/


\newif\ifInputs
\Inputstrue
%\Inputsfalse

%\usepackage[sort,compress]{cite} % this doesn't work with elsevier styles
\usepackage{palatino}
\usepackage[usenames]{color}
\usepackage{epsfig}
\usepackage[fleqn,reqno]{amsmath}
\usepackage{amsfonts,amsthm,bm}
\usepackage{amssymb}
\usepackage{mathrsfs}
\usepackage{stmaryrd}
%\usepackage{subfigure, appendix}
\usepackage{subfigure}
\usepackage[titletoc]{appendix}
\usepackage{tabularx,booktabs}
\usepackage{graphics,caption}
\usepackage{fancybox}
\usepackage[top=1.2in,bottom=1.2in,left=1in, right=1in]{geometry}
\usepackage{array}
\usepackage{multirow}
\usepackage{wrapfig}
\usepackage{algorithmic,algorithm}
\usepackage{paralist}
\usepackage{ifthen}
\usepackage{enumitem}
\usepackage{tikz,pgfplots,filecontents}
\usepackage{mdframed}
%\usepackage{showkeys}
%To see the labels for now.  Will remove later

%%%%%%  pdftex  %%%%%%%%%%%%%%%%%%%%%%%%%%%%%%%%%%%%%%%%%%%%%%%%%%%%%%
\usepackage[pagebackref=false,bookmarks=false]{hyperref} 

\definecolor{dblue}{rgb}{0.03,0.3,0.62}
\definecolor{dorange}{rgb}{1,0.55,0}

\hypersetup{
  bookmarksnumbered=true,
  bookmarksopen=false,
  hypertexnames=false,      
  breaklinks=true,          
  unicode=false,
  pdffitwindow=true,        
  pdfnewwindow=true,        
  colorlinks=true,         
  linkcolor=dblue,
  anchorcolor=red,
  citecolor=dorange,
  filecolor=magenta,
  urlcolor=dblue,
  pdfstartview = FitH,
  %pdftitle = {},
  %pdfsubject = {Subject},
  pdfkeywords = {},
  pdfcreator = {LaTeX with hyperref package}
}

%%%%%%%%%%%%%%%%%%%%%%%%%%%%%%%%%%%%%%%%%%%%%%%%%%%%%%%%%%%%%%%%%%%%%%
%\renewcommand{\algorithmiccomment}[1]{
%        \settowidth\textLen{\footnotesize\tt// #1}
%        \setlength\resLen{\the\commLen}
%        \addtolength\resLen{-\the\textLen} 
%        \hfill{\footnotesize\tt// #1}\hspace{\the\resLen}}

\definecolor{sblue}{cmyk}{0.98,0.13,0,0.43} % section blue
\definecolor{sblue}{cmyk}{0.98,0.13,0,0.43} % section blue
\newcommand{\csblue} {\color{sblue}}  %\renewcommand{\csblue}{\color{black}}
\newcommand{\cblack} {\color{black}}
\newcommand{\sblue}[1]{{\color{sblue} #1}}

\newcommand{\todo}[1]{ \fbox{{\bf TODO:} \color{red} #1}}
\newcommand{\comment}[1]{\sblue{#1}}
\def\gap{\hspace*{.2in}}
\newcommand{\half}[1]{\frac{#1}{2}}
\newcommand{\bigO}{\mathcal{O}}
\newcommand{\Div}{\mathrm{Div}}
\newcommand{\p}{\partial}
\newcommand{\avg}[1]{\left< #1 \right>}


%Bryan's new commands
\newcommand{\xx}{{\mathbf{x}}}
\newcommand{\yy}{{\mathbf{y}}}
\newcommand{\ff}{{\mathbf{f}}}
\renewcommand{\SS}{{\mathcal{S}}}
\newcommand{\BB}{{\mathcal{B}}}
\newcommand{\DD}{{\mathcal{D}}}
\newcommand{\EE}{{\mathcal{E}}}
\newcommand{\TT}{{\mathcal{T}}}
\newcommand{\pderiv}[2]{\frac{\partial #1}{\partial #2}}
\newcommand{\cc}{{\mathbf{c}}}
\newcommand{\uu}{{\mathbf{u}}}
\newcommand{\UU}{{\mathbf{U}}}
\newcommand{\vv}{{\mathbf{v}}}
\newcommand{\rr}{{\mathbf{r}}}
\newcommand{\nn}{{\mathbf{n}}}
\newcommand{\adh}{{\mathrm{adh}}}
\newcommand{\new}{{\mathrm{new}}}
\newcommand{\old}{{\mathrm{old}}}
\newcommand{\eeta}{{\boldsymbol\eta}}
\newcommand{\zzeta}{{\boldsymbol\zeta}}
\newcommand{\ttau}{{\boldsymbol\tau}}
\newcommand{\ssigma}{{\boldsymbol\sigma}}
\newcommand{\llambda}{{\boldsymbol\lambda}}
\newcommand{\grad}{{\triangledown}}
\DeclareMathOperator*{\argmin}{\arg\!\min}
%\renewcommand{\algorithmiccomment}[1]{\hfill \texttt{\% #1}}
\renewcommand{\algorithmiccomment}[1]{\hfill{\footnotesize\tt #1}}


\newcolumntype{C}{>{\centering\arraybackslash} m{2.5cm}}

\usepackage{lineno}

\newcommand{\mcaption}[2]{\caption{\small \em #1}\label{#2}}
\newcommand{\secref}[1]{\ref{#1}}

\begin{document}

\title{High-volume fraction simulations of two-dimensional vesicle
  suspensions}

\author[ut]{Bryan Quaife} \ead{quaife@ices.utexas.edu}
\author[ut]{George Biros}\ead{gbiros@acm.org}
\address[ut]{Institute of Computational Engineering and Sciences,\\
  The University of Texas at Austin, Austin, TX, 78712.}

\begin{abstract} 
We consider numerical algorithms for the simulation of the rheology of
two-dimensional vesicles suspended in a viscous Stokesian fluid.  The
vesicle evolution dynamics is governed by hydrodynamic and elastic
forces. The elastic forces are due to local inextensibility of the
vesicle membrane and resistance to bending.  Numerically resolving
vesicle flows poses several challenges.  For example, we need to
resolve moving interfaces, address stiffness due to bending, enforce
the inextensibility constraint, and efficiently compute the
(non-negligible) long-range hydrodynamic interactions.

Our method is based on the work of {\em Rahimian, Veerapaneni, and
Biros, ``Dynamic simulation of locally inextensible vesicles suspended
in an arbitrary two-dimensional domain, a boundary integral method'',
Journal of Computational Physics, 229 (18), 2010}.  It is a boundary
integral formulation of the Stokes equations coupled to the interface
mass continuity and force balance.  We extend the algorithms presented
in that paper to increase the robustness of the method and enable
simulations with concentrated suspensions.

In particular, we propose a scheme in which both intra-vesicle and
inter-vesicle interactions are treated semi-implicitly.  In addition we
use special integration for near-singular integrals and we introduce a
spectrally accurate collision detection scheme.  We test the proposed
methodologies on both unconfined and confined flows for vesicles whose
internal fluid may have a viscosity contrast with the bulk medium.  Our
experiments demonstrate the importance of treating both intra-vesicle
and inter-vesicle interactions accurately.
\end{abstract}

\begin{keyword}
  Stokes flow \sep Suspensions \sep Particulate flows \sep Vesicle
  simulations \sep Boundary integral method \sep Fluid
  membranes \sep   Semi-implicit algorithms \sep Fluid-structure
  interaction \sep Spectral collision detection \sep 
  Fast multipole methods 
\end{keyword}

\maketitle

\section{Introduction\label{s:intro}}
\input introduction.tex

\section{Formulation\label{s:formulation}} 
\input formulation.tex

\section{Method\label{s:method}} 
\input method.tex

\section{Computing Local Averages of Pressure and Stress\label{s:aver}}
\input aver.tex

\section{Results\label{s:results}} 
\input results.tex

\section{Conclusions\label{s:conclusions}}
\input conclusions.tex


%\begin{a2dnearppendices}
\begin{appendices}
\section{Error estimates for near-singular integration \label{A:AppendixA}} 
\input appen1.tex

\section{Jumps in pressure and stress \label{A:AppendixB}} 
\input appen2.tex

\section{Variable curvature formulation \label{A:AppendixC}}
\input appen3.tex
\end{appendices}


\bibliographystyle{plainnat} 
\bibliography{refs}
\biboptions{sort&compress}
\end{document}
