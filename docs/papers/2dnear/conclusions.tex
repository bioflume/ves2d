We have presented and tested a collection of extensions of the boundary
integral equation formulation for vesicle suspensions outlined
in~\cite{rah:vee:bir}.  Our goal is to create a robust method for
handling high concentration suspensions.  The main contributions we have
presented are:
\begin{itemize}

\item To remove stiffness, we have introduced a new time integrator that
treats inter-vesicle interactions semi-implicitly.  This allows us to
take larger time steps and to use second-order time stepping.  

\item To handle vesicles in close proximity, we employed and tested the
near-singular integration algorithm outlined
in~\cite{ying-biros-zorin06}.  This algorithm creates a uniform error
when evaluating integral operators, is easy to implement, and does not
significantly increase the overall complexity.  The vesicle flow
configurations that we tested are unstable without near-singular
integration.

\item To test for collisions, we introduced a spectrally accurate
algorithm that uses standard potential theory and is compatible with
our near-singular integration scheme and the fast multipole method.

\item We computed the pressure and stress of the single- and
double-layer potentials.  To use near-singular integration, we require
the limiting values of the pressure and stress as the target point
approaches the boundary.  All these jumps were computed.

\item We obtained first- and second-order convergence for a variety of
bounded and unbounded flows.  We also observed examples where the errors
for second-order convergence plateau while the error for first-order
convergence continues to grow.
\end{itemize}

While these contributions are a major step towards creating a robust
solver for high concentration vesicle suspensions, there are several key
features that are necessary and are currently being investigated.

\begin{itemize}

\item The vesicle shapes we have presented in Section~\ref{s:results}
can be well represented with $N=128$ or fewer points per vesicle.
However, we may require more points to represent quantities such as the
traction jump or the velocity field.  Therefore, spatial adaptivity is 
under consideration.

\item Currently, the time step is found using a trial and error
process.  We are developing high-order time integrators that use error
control to automatically and adaptively adjust the time step size.

\item Our formulation is two-dimensional.  The algorithms we have
introduced naturally extend to three-dimensions, but the linear systems
are far too expensive to solve without suitable preconditioners. 

\item Our formulation uses the steady Stokes equations for both the
fluid in the bulk and in the vesicle interior. If transient, inertial,
or viscoelastic effects are important one has to use a different
formulation.

\end{itemize}


%\todo{acknowledgments of funding.}

