\documentclass[12pt]{article}

\usepackage{fullpage}
\usepackage{amsmath,amsfonts,amssymb,stmaryrd}
\usepackage{color}
\usepackage{enumitem}
\newcommand{\comment}[1]{{\color{blue} #1}}
\newcommand{\todo}[1]{{\color{red} #1}}
\newcommand{\note}[1]{{\color{green} #1}}
\newcommand{\grad}{{\triangledown}}
\newcommand{\nn}{{\mathbf{n}}}
\newcommand{\sdc}{{\mathrm{sdc}}}
\newcommand{\uu}{{\mathbf{u}}}
\newcommand{\xx}{{\mathbf{x}}}
\newcommand{\yy}{{\mathbf{y}}}
\newcommand{\DD}{{\mathcal{D}}}
\renewcommand{\SS}{{\mathcal{S}}}

\title{Revisions to: Adaptive Time Stepping for \\ Vesicle Suspensions}
\author{Bryan Quaife and George Biros}

\begin{document}

\maketitle


%%%%%%%%%%%%%%%%%%%%%%%%%%%%%%%%%%%%%%%%%%%%%%%%%%%%%%%%%%%%%%%%%%%%%%%%
\section*{Significance of work}
In this paper, we propose a time integrator for vesicle suspensions
that 1) is adaptive in order to enable black-box use of the solver by
non-experts and avoid the need to manually select the time-step size;
2) is at least second-order accurate; 3) addresses stiffness; and 4)
can be applied with reasonable computational costs.

Adaptivity is required for vesicle suspensions, especially when we have
multiple time scales during the course of a simulation. Such multiple
scales make estimation of the correct time step cumbersome and
expensive.  Second- or higher-order is required since we are interested
in simulations with long time horizons and we have found that standard
first-order methods require excessively small time steps.

We propose a novel time integrator that is specific to vesicle flows.
It is a spectral deferred correction (SDC) method that uses IMEX Euler
at each SDC iteration.  For the selection of the time step, we propose
a scheme that exploits the invariance of the vesicle area and length to
estimate the error very cheaply. We accelerate the Krylov matvecs with
the fast multipole method, and we propose a block-diagonal
preconditioner that is frozen at the start of each time step. 

To the best of our knowledge, an adaptive, stable, second-order (or
higher) accurate scheme for vesicle suspensions is novel and has not
been reported before in the literature.

To analyze the performance of our scheme in practical situations, we
compared it to the previous state-of-the-art time integrator for
vesicle simulations, our second-order time integrator, BDF, that can
only use a {\em fixed time step}.  For the {\em extensional} and {\em
stenosis} flows, our new method is much faster than BDF, even if we
neglect the costs of selecting the time step for the BDF scheme.  For
the {\em Couette} examples (42 and 150 vesicles), BDF is comparable to
our scheme, again assuming we know the right time-step for the BDF a
priori. Again our main point is that the adaptive scheme alleviates the
need for time step selection. In the new scheme a user only needs to
specify the target accuracy.


%%%%%%%%%%%%%%%%%%%%%%%%%%%%%%%%%%%%%%%%%%%%%%%%%%%%%%%%%%%%%%%%%%%%%%%%
\section*{Major changes in this revision}
Here we provide an itemized summary of the major changes between the
previous and current version of the manuscript. When a table number has
been changed between the current and previous submissions, we place the
index from the previous version in parenthesis.

\begin{itemize}
  \item {\em Recomputed the tables.}  Following the suggestion of one
  of the reviewers, in the {\em Results} section, we took the smallest
  number of substeps that guarantees that the quadrature order of
  accuracy exceeds $n_{\sdc}+1$. The results in Tables 3, 4, 6, 8, 9,
  10 (13) are new, and Tables 12 and 14 are entirely new calculations.

  \item {\em The stenosis example has been modified.}  We replaced the
  old {\em stenosis} example with a new example in which we increased
  the length of the channel without changing the size of the
  constriction (see Figure 5) so that the effects of long-time horizon
  and different dynamics can be better illustrated.  From our previous
  submission, we removed Tables (10), (11), (12), and (14), and instead
  did a more in depth comparison of the two second-order time
  integrators.  That is, we compared BDF (this run was absent in the
  previous submission) with adaptive $n_{\sdc}=1$.  For BDF, we plot
  the errors in Figure 6 and it is clear that there is a sharp jump in
  the error as the vesicle passes through the constriction.  In Table
  10 (13), we show the results of our new adaptive time integrator and
  the time step size for a single run is in the left plot of Figure 7.
  It shows that for this example, the time step size varies over three
  orders of magnitude.  Finally, in the right plot of Figure 8, we
  compare the errors of the two time integrators as a function of the
  CPU time. Our new time integrator is about an order of magnitude
  faster than BDF.

%We plotted the error at the time horizon as a function of the CPU
%  time (see Figure 7) for two second-order time integrator: 1) BDF with
%  a constant time step size; and 2) $n_{\sdc}=1$ with an adaptive time
%  step size. Our new time integrator outperforms BDF.
  

%  \todo{Why you removed them? You need to explain.}
%  We also ran BDF for this example (this was
%  absent in the previous submission), and the errors are illustrated in
%  Figures 6 and 7 (right). 
%  \todo{Bryan,  the sentence below will cause more criticism and
%  rejection of the paper ``full convergence study is in not
%  important''!?  Please rephrase.}
% We felt that doing a full convergence study
%  was not as important as showing that our new time integrator is faster
%  than BDF.
%\todo{For example: In Table X we demonstrated
% convergence of the method. With Figures 6 and 7 compare our scheme
% to the second order BDF, which is the state of the art.}
%  Figure 6 now shows the error of BDF as a function of time.
%  It shows how the error does increase rapidly as the vesicle passes
%  through the constriction.

  \item {\em New figures to summarize results.} For the {\em
  extensional} and {\em stenosis} examples, we have added plots of the
  error versus CPU time for the second-order time integrators (see the
  right plots of Figures 4 and 7).  These numbers are presented in the
  appropriate tables, but this allows the readers to quickly compare
  the different methods.


  \item {\em Two new results for the Couette apparatus.}  The {\em
  Couette} example from the previous submission is completely removed.
  We replaced it with two runs (42 vesicles and 150 vesicles), both of
  which were presented in our previous JCP paper, following the
  suggestion of the reviewers.  We compare our old and new second-order
  time integrators (BDF was absent from the previous submission).  We
  have emphasized how one has to often use a trial-and-error procedure
  with BDF to find a desired time step size.  The two schemes are
  comparable, but with BDF, a trial-and-error needs to take place to
  select the time step, which may be too expensive. Please see the
  {\em Results} section for more discussion.
  %% GB:too much detail leave it for 
  %% Assuming one is able to somehow pick the correct time
  %% step size for BDF, the two second-order time integrators are
  %% comparable for 42 vesicles, and BDF is twice as fast for 150 vesicles.
  %% However, we see that if an appropriate time step size is not chosen
  %% for BDF, the tolerance may be exceeded.  In particular, for the 150
  %% vesicle simulation, we see that this can happen near the time horizon
  %% ($m=900$), at which point this simulation becomes useless.


  \item {\em Text modifications to clarify issues with BDF.} We have
  emphasized throughout the manuscript that BDF is not compatible with
  an adaptive time step size, whereas our new time integrator is.  The
  result is a much more robust method for choosing the time step size.
  The time integrator always tries to commit the same amount of error
  per time step, is second-order accurate, and efficiently estimates
  the local truncation error.

  \item {\em Text modifications to clarify SDC.} Following the
  suggestion of the reviewers, we have described SDC as an iterative
  method that converges to the solution of the fully-implicit Gaussian
  scheme.  We have used this interpretation to better describe order
  reduction, and we added several new references which address order
  reduction, including Krylov deferred correction, multilevel SDC,
  PFASST, Parareal, and DIRK (see references [7, 8, 16, 17, 24, 25, 26,
  27, 32, 46, 47, 48, 53, 54]).

%% un important
%%   \item In the conclusions, we discuss multirate time integrators.  We
%%   see in our new {\em couette} examples that for high-concentration
%%   suspensions, allowing each vesicle to have its own time step size
%%   could allow for significant speedups.


\end{itemize}

Several other changes that the reviewers discussed are described in
the sections below.



%%%%%%%%%%%%%%%%%%%%%%%%%%%%%%%%%%%%%%%%%%%%%%%%%%%%%%%%%%%%%%%%%%%%%%%%
\section*{Response to editor}
\comment{For the article to be published, it has to be clear that the
new method is really an improvement at least compared to your own
previous BDF based fixed dt method that you can very easily compare to,
and also that the time step selection is effective}

We have compared BDF to our new second-order adaptive time step time
integrator in all of the numerical examples.  For the extensional
example and relaxation (constant time step size), it is clear that our
new time integrator is faster than BDF.  This is mentioned in the
text.  For the {\em stenosis} flow, we increased the length of the
geometry so that adaptive time stepping is more critical.  For this
example, our new time integrator is about an order of magnitude more
accurate than BDF. More important is {\bf robustness}. With the new
scheme, the time step selection is automatic and the solver can be used
in black-box fashion.   We have made this clear throughout the
manuscript.

We have removed the previously submitted {\em Couette} example and
substituted it with the two Couette examples from our previous paper.
We have demonstrated how BDF fails if too large of a time step is
chosen.  In the suspension with 150 vesicles, we show how BDF can fail
at 80\% of the time horizon, at which point all the CPU time used is
rendered useless.  For the 42 vesicle simulation, the errors achieved by
BDF and our new time integrator are nearly identical.  However, our time
integrator has the advantage that the user does not have to prescribe
the time step size, only the error tolerance. 
%For the 150 vesicle simulation, BDF is about twice as fast as our new
%time integrator.  Again, this is assuming that the user knows the time
%step size that BDF should take to achieve the desired tolerance.  This
%example also demonstrates that our method would benefit from a
%multirate time integrator, where each vesicle has its own time step
%size.  This is discussed in the {\em Results} and {\em Conclusions}
%sections.
\\ \\
\comment{I do not understand your reply to Reviewer 1 on a comment on
the original manuscript. Here he asks you to show results for examples
in your previous work to see that you actually get a reduction in
computational time. Your answer talks about the examples not being
robust and the error behaving unpredictably. But these are indeed the
kind of problems that you want to use your new method for? I think it
seems most reasonable that you present results for the problems that you
have considered with your previous method.}

The lack or robustness was referring to the old method, not the new
one. The second-order time integrator BDF requires a trial-and-error
procedure is required to find a time step size that guarantees that
collisions will not occur, and that a desired tolerance can be
achieved. The trial-and-error is required because BDF is not compatible
with an adaptive time step size which is necessary for flows such as
the ones presented in the manuscript. We now present results for the
same problems we considered with the previous method in our previous
JCP paper.


%%%%%%%%%%%%%%%%%%%%%%%%%%%%%%%%%%%%%%%%%%%%%%%%%%%%%%%%%%%%%%%%%%%%%%%%
\section*{Response to reviewer 1}

\comment{While all suggestions have been commented on, almost non is
truly addressed. To design a high-order adaptive scheme in order to
have such a scheme, without demonstrating an improvement to existing
schemes, does not qualify for publication in JCP.}

We have made significant changes to the {\em Results} section.  In
particular, we do all the runs from our previous paper.  A more
thorough comparison of our old and new methods is presented.  The main
improvement is that {\em the time step is selected automatically}. With
BDF, we always had to perform several runs to figure out the right time
step. Now this is not necessary and overall the scheme is faster.
Therefore, one should not simply compare the final errors and CPU times
in the current manuscript; one should also account for the fact that
the time step size selection is automatic for our new method, and not
for BDF. But even if we do not account for the trial-and-error costs of
the BDF, for several examples, SDC is plainly faster than BDF,
especially when the flow problem has multiple time scales (e.g., {\em
stenosis} and {\em extensional} flow).

% In the {\em couette} example we
%report runs in which the BDF
%is either unstable or exceeds the specified ttolerance of $1$E$-1$
%because too large of a time step size is chosen.


%%%%%%%%%%%%%%%%%%%%%%%%%%%%%%%%%%%%%%%%%%%%%%%%%%%%%%%%%%%%%%%%%%%%%%%%
\section*{Response to reviewer 2}
\comment{I think it is important to separate the cause of order
reduction from the Krylov method in [21]. Order reduction happens
because the solution has not converged to the collocation scheme. This
could be remedied by simply doing many SDC iterations until convergence
(as in [47] for stiff problems).  The Krylov method in [21] is simply a
way of accelerating that convergence, but there are other options
including recent multigrid-type ideas from Speck and collaborators.
While the revisions have improved the discussion of order reduction,
the text now suggests that one must either have order reduction or use
the method in [21] which is not the case. The choice made here is to
used a fixed number of iterations and try to manipulate the time step
to meet an error criteria. The alternative is to use more iterations
(or acceleration) to improve the error per step. The fixed iteration
choice is defensible given the results, but the text inappropriately
implies that the only other option is to use the Krylov method. The
question of which is more efficient ( more time steps with fewer
iterations or fewer time steps with more iterations) is a much more
difficult problem.  This should be made clear.}

We have made numerous changes in the Sections {\em Related work,
Limitations, Spectral Deferred Corrections}, and {\em Adaptive Time
Stepping}.  We tried to do a better job of describing why order
reduction occurs, as well as clarified some of its remedies discussed
in the literature.  We have cited work of Speck and collaborators,
Falgout et al., and kept the Krylov deferred correction citations in
place.  All of the newly cited works come from the multilayer SDC,
multigrid in time, and PFASST literature.  However, seeing as SDC has
not previously been tested on vesicle suspensions, we have opted to
simply choose a fixed number of corrections, an appropriate quadrature
formula, and then apply SDC. \\ \\
\comment{This is exactly the point I was making. Given your statement,
why are you choosing $p$ so that the quadrature is so accurate? The
error is generated not by the quadrature rule, but by the fact that the
thing being integrated is not accurate.  The footnote on page 19 adds
confusion "Recall that $p-1=3$ additional time steps are required
because of the intermediate Gauss-Lobatto quadrature points".  $p-1=3$
is the number of additional substeps in the SDC formulation and not the
number of time steps (or I am really confused).}

We have redone the {\em Results} section and used the smallest value of
$p$ such that $2p-3 \geq n_{\sdc}+1$.  This improved some of the fixed
time step size results.  However, when doing adaptive time stepping,
because of the larger substeps, smaller time steps need to be taken to
guarantee the desired accuracy.  Therefore, there are additional
accepted and rejected time steps in some of the adaptive runs,
especially those with larger tolerances.
\\ \\
\comment{This makes no sense. Clearly if the adaptive method was forced
to never take a time step finer than the fixed time step, the error
should be the same in the constriction. If a larger time step can be
used elsewhere, then there is a savings. The authors obviously have a
lousy time step selector for this configuration. It is a first order
"straw man" argument used to make the second order results seem more
impressive.}

Our previous statement was incorrect.  We were not comparing appropriate
fixed and adaptive runs, and we apologize for the confusion.  After
contemplating remarks from the editor and the other reviewer, we have
opted to only compare BDF with adaptive $n_{\sdc}=1$.  First-order time
stepping is suboptimal and we think it clutters the discussion.  When
comparing these two second-order time integrators, for this example, we
are able to achieve about an extra digit of accuracy in the same amount
of CPU time.

\end{document}
