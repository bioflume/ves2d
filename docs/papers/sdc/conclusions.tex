We have developed and tested an adaptive time stepping method for an
integro-differential equation formulation for the numerical simulation
of vesicle suspensions.  Adaptive time stepping alleviates the need for
time step size selection which increases robustness, increases the
performance in the presence of time scales that vary significantly
throughout the simulation, and can be used to deliver any final-time
accuracy in a controlled way.

To select the time step size, we followed the standard strategy of
estimating the local truncation error and then adjusting the time step
size so that the same amount of error is committed per time step.
Standard methods require the computation of more accurate provisional
solutions, which can be very expensive in our context. Instead, we
introduced a technique where the error in area and length of the
vesicles, which are invariant for vesicles, are used to estimate the
local truncation error---basically for free.  We justified the use of
these invariants by (empirically) showing that the error in position
decays at the same rate as the errors in area and length.  While this
method proved to be reliable, it can only be applied to problems with
measurable invariants. But many other particulate flows share similar
properties; for example, any non-permeable capsule filled with an
incompressible fluid must preserve its volume.

To allow for larger time steps and longer time horizons, we
introduced new high-order time integrators based on spectral deferred
corrections (SDC).  An advantage of these integrators is that, in
contrast to our previous second-order integrator, BDF, they are
compatible with an adaptive time step size.  This time integrator is
accelerated using the fast multipole method and a preconditioner that
is factored once per time step, and then frozen for all the
Gauss-Lobatto substeps and SDC iterations.  To the best of our
knowledge, this work is the first to apply IMEX methods coupled with
SDC to a two-dimensional integro-differential equation.

These contributions are a major step towards the development of a robust
solver for vesicle suspensions, but several other features need to be
introduced.
\begin{itemize}
  \item Theoretical results, in particular the convergence rates of
  SDC, need to be developed.  This will allow us to take the largest
  possible time step size given the user specified tolerance.  Since
  the governing equations for vesicle suspension are very stiff, order
  reduction is expected and observed.  There are methods to remove
  order reduction such as Krylov deferred
  correction~\cite{hua:jia:min2006, hua:jia:min2007, bu:hua:min2012,
  bu:hua:min2009, jia:hua2008} and multilevel SDC~\cite{emm:min2012,
  min:spe:bol:emm:rup2015, spe:rup:emm:min:bol:kra2014}.  Another
  possible direction to accelerate the method is to use inexact
  SDC~\cite{spe:rup:min:emm:kra2014, spe:rup:emm:min:bol:kra2014,
  min:spe:bol:emm:rup2015}.  To the best of our knowledge, none of
  these methods have been applied to intego-differential equations,
  such as those that govern vesicle suspensions.

  \item Many vesicle suspensions could benefit from a time integrator
  that treats each vesicle with its own time step size, or a multirate
  time integrator.  This is apparent in the {\em Couette} example where
  the actual error was not very tight to the desired tolerance.

  \item All of our results are two-dimensional.  However, none of the
  methods that we have introduced in this paper are restricted to two
  dimensions.  In particular, the volume and a surface area of a
  three-dimensional vesicle can be accurately computed using its
  spherical harmonic representation~\cite{vee:rah:bir:zor2011}.
  Performing three-dimensional simulations only requires a careful
  implementation of the methods and the development of suitable
  preconditioners.

\end{itemize}
