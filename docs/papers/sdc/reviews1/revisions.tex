\documentclass[12pt]{article}

\usepackage{fullpage}
\usepackage{amsmath,amsfonts,amssymb,stmaryrd}
\usepackage{color}
\newcommand{\comment}[1]{{\color{blue} #1}}
\newcommand{\todo}[1]{{\color{red} #1}}
\newcommand{\note}[1]{{\color{green} #1}}
\newcommand{\grad}{{\triangledown}}
\newcommand{\nn}{{\mathbf{n}}}
\newcommand{\uu}{{\mathbf{u}}}
\newcommand{\xx}{{\mathbf{x}}}
\newcommand{\yy}{{\mathbf{y}}}
\newcommand{\DD}{{\mathcal{D}}}
\renewcommand{\SS}{{\mathcal{S}}}

\begin{document}

We would like to thank both reviewers for their careful reading of our
manuscript and for their constructive feedback. 

\section*{Reviewer 1}

\comment{several parts of the introduction are more or less copied from
the earlier paper}

\begin{itemize}
  \item While we agree that some statements are similar or identical,
  the use of those statements are justified for both papers as they are
  related to the general problem of vesicle suspension simulations. For
  example using text diff,  the two instances that we found to be
  identical or most similar are: 
  \begin{enumerate}
    \item Vesicles are deformable capsules filled with a viscous fluid
    \item For example, they are used experimentally to understand
    properties of biomembranes [36] and red blood cells [8, 13,
    24-26] \emph{vs.} Vesicle flows find many applications many biological
    applications such as the simulation of biomembranes [39] and red
    blood cells [16, 25, 30, 31, 33].
  \end{enumerate}
\end{itemize}
These are generic statements that motivate the problem and, in our
opinion, are not related to the novelty of our contribution.

\comment{the paper is hard to read as a stand alone paper, as even the
system of equations to be solved is not written down}
\begin{itemize}
  \item In Section 2.2 we state the governing equations. We agree that
    the paper is somewhat hard as we have not included the derivation
    of these equations, but we have tried to make sure that all
    necessary information is included to replicate our results.
\end{itemize}

\comment{it is nice to see a statement of limitations of the method,
however it is not understandable, why several limitations mentioned in
the earlier paper, should have gone away, e.g.  transient and inertia
effects}

\begin{itemize}
  \item The reviewer is correct that not all the limitations from our
    previous paper have been addressed in this manuscript.  In fact,
    we have only addressed one of the issues---high-order time
    discretization and time adaptivity. We have made this clearer in
    the text.
\end{itemize}

\comment{what is highlighted here as an innovative criteria for adaptive
time stepping, is just a phenomenological indicator, it is certainly
good to control area and interface length, but it does not say anything
about the quality of the solution (a constant in time solution would do
the job, but would be wrong for most examples). Here I miss
computational results, demonstrating that the used error indicators are
indeed sufficient to control the error of the solution. At least for the
relaxation flow the stationary shape and the evolution to it are known
and can be used as a criteria.  Otherwise you could use criteria such as
circularity or center of mass position to validate your solution. Can
you show that the energy is decreasing for all the chosen time steps?}

\begin{itemize}
  \item The new adaptivity is not our only contribution. We present
  non-trivial SDC approximations that are specific to vesicle flows and
  we demonstrate their effectiveness. 

  \item We agree that the area and length are phenomenological
  indicators.  We introduced them because they are \emph{so cheap} to
  compute.  Our assumption is that the errors in area and length behave
  similarly as the true error and therefore using them as an estimate
  of the local truncation error is appropriate. Of course, the reviewer
  is right to request at least empirical evidence that this is the
  case.  For the relaxation flow, we have included an additional column
  that computes the error in the vesicle shape at the time horizon.  We
  computed an ``exact" solution using 4000 time steps with 4 SDC
  corrections.  The error in position behaves the same as the errors in
  length and area, that confirms our assumption.

  \item In our previous paper [34], we showed convergence of the
  vesicle-wall distance for the stenosis example (Table 9), again
  providing experimental evidence  that when the error and area errors
  are small the solution has converged. 

  \item In the current manuscript, we discuss in Section 4 that we can
  justify using the error in area and length as an indicator for the
  local truncation error since we do a proper convergence study for the
  vesicle position in Section 5.
\end{itemize}

\comment{the comparison in 5.1 remains unclear, is the BDF the second
order scheme from your previous paper or is it a different method, how
does SDC compare with the results in the previous paper? As already
mentioned in the text an improvement of SDC can not really be observed
for the second order scheme}

\begin{itemize}
  \item We have mentioned that BDF is the multistep second-order method
  we introduced in our previous paper. The improvement is not that SDC
  is faster, but that it is adaptive.  Our goal in this paper is the
  design of high-order, {\em adaptive} methods.  With BDF one must use a
  fixed time step size.  BDF has the hidden cost of iterating (running
  extra simulations) to find out the stable and accurate time step size.
  This is not the case with adaptive SDC.  We have made this clearer in
  the text.
\end{itemize}

\comment{as almost the same examples are used as in the previous paper,
a follow up paper should at least used the same complexity of the
examples and demonstrate an improvement in computing time, much better
would be to show an improvement in complexity of the problem solvable,
in 5.4 the complexity is even reduced, which is not acceptable}

\begin{itemize}
  \item The numerical examples that we presented in the previous paper
  are not robust since they use fixed time stepping.  We had to run many
  simulations for them to find the right step size.  For instance, in
  Figures 5, 6, and 8 from our previous paper, [34], we showed that  the
  error in area and length may behave unpredictably.  However, in the
  right plots of Figures 4, 7, and 10 from our current manuscript have
  errors that behave as desired.
\end{itemize}

\comment{it is the second algorithmic paper of the authors on the
subject, any meaningful biophysical question, which could be answered by
the simulations, is not even asked, and should be added and answered in
a revised version}
\begin{itemize}
  \item The goal of this paper is to develop an additional tool, namely
  high-order adaptive time stepping, for simulating vesicle suspensions.
  This is a very challenging numerical scheme and we think there is a
  lot of work that remains to be done (algorithmically).  For this
  reason, we have chosen to submit the paper the Journal of
  Computational Physics.  We are now using these methods to study
  meaningful biophysical questions which will be submitted elsewhere.
  Note that we (and other groups) have used our code for such
  investigations, for example see [16,25,30].
\end{itemize}




%%%%%%%%%%%%%%%%%%%%%%%%%%%%%%%%%%%%%%%%%%%%%%%%%%%%%%%%%%%%%%%%%%%%%%%%
\section*{Reviewer 2}
\comment{I think the presentation in this paper generates some
confusion on the issue of the accuracy of SDC in the stiff regime. The
paper does report on the early results on order reduction (e.g. [26]),
but the issue is much better understood now than when [26] was written.
In [18] for example, the authors show that the order-reduction
previously observed for SDC for stiff problems is equivalent to the slow
convergence of the numerical solution to that given by a fully implicit
Gaussian collocation scheme (or equivalently a fully implicit
Gauss-Runge-Kutta method). This subtle shift from viewing SDC as a
method using a fixed number of iterations to achieve a fixed order of
accuracy to viewing SDC as an iterative procedure that converges to a
spectrally accurate collocation method is really the key to
understanding the behavior for stiff problems.}
\begin{itemize}
  \item We now mention in the introduction that order reduction is
  observed in [29] (used to be [26]) and that this order reduction was
  better understood and resolved in [21] (used to be [18]).  This is
  now discussed at multiple points locations in the manuscript.

  \item The challenge with using methods described in [21] is that an
  entire history of the iterates must be saved since SDC is being used
  as a preconditioner of a GMRES iteration.  Moreover, there is an outer
  Newton iteration since the governing equations are non-linear.  At
  this point in our project, we do not want to bring in this additional
  algorithmic complexity into the method.  We estimate that such a
  modification will slow down the scheme significantly since a Newton
  method will require globalization.  In our past work we have compared
  Newton method with semi-implicit methods (using simple BDF) and we
  found that the Newton method was excessively slower for the same
  accuracy.
\end{itemize}

\comment{Although Thms like the 2.1 cited here show that the formal
order of accuracy of SDC increases in lock-step with the number of
correction iterations, these results are only observed asymptotically as
the time step approaches zero. For stiff problems, the iterations for
explicit SDC methods will diverge in $k$ for sufficiently large time
step.  For fully implicit methods the convergence of SDC iterations in
$k$ will slow down, and order reduction is observed. Hence there are two
``convergences" going on: the convergence in iteration $k$ where the SDC
solution approaches the collocations solution and the convergence in
$\Delta t$ as the numerical solution approaches the exact solution. The
citation [44] examines the first convergence carefully for linear
problems. If the simulations are always done with enough SDC iterations
so that the iterations have converged in $k$ (or some sort of
acceleration of the convergence is used), then the observed convergence
rate will be that of the collocation scheme (which is not necessarily
the order in the non-stiff case. See for example Hairer and Wanner II).
If a fixed number of iterations are used, the asymptotic rate may not be
observed until $\Delta t$ is sufficiently small, i.e. order reduction is
seen.}
\begin{itemize}
  \item We agree. We have changed the text and we discuss these issues
  throughout the text.
\end{itemize}

\comment{Unfortunately, the numerical results presented here mix these
issues up.  For the test in 5.1, $p=5$ nodes are used, so that if many
SDC iterations are used, the observed rate should be 8. For a fixed
number of iterations, order $k$ convergence should be observed for
sufficiently small $\Delta t$, but this is obscured because of machine
precision (and GMRES precision). It is not really clear why one would
choose $p=5$ nodes and 2, 3 or 4 iterations anyway due to the fact that
it requires more storage for the same order of accuracy as using fewer
nodes. Similar issues are appropriate for other tests.}
\begin{itemize}
  \item We agree that smaller values of $p$ can be used for many of the
  reported results.  Unfortunately, if $p$ is decreased, the time step
  size is increased, and the number of GMRES iterations to solve the
  implicit system increases.  We want to avoid this increase in the
  number of GMRES iterations so that we can compare the total number of
  FMM calls required for different time integrators.
  \item In order to guarantee that the quadrature error has a
  negligible effect on the overall error, we took $p$ sufficiently
  large that it has an additional three orders of accuracy than the
  most accurate time integrator used in each example.  This amounts to
  using $p=5$ for the relaxation flow (the most accurate runs have
  fifth-order accuracy), and $p=4$ for the remaining flows (the most
  accurate runs have fourth-order accuracy).
  \item We discuss this justification at the start of the Results
  section.  We also mention that if one is not doing a convergence
  study/comparison of different methods, then smaller values of $p$
  can be used without compromising the overall accuracy.
\end{itemize}


\comment{So in summary, the use SDC methods for stiff problems leads to
a very subtle trade off. Using enough iterations so that the SDC
iterations converge (which can be checked by monitoring the residual)
requires more work, but insures the optimal order of accuracy and
stability of the Gauss collocation schemes. Using a fewer fixed number
means that the order for stiff problems will be reduced. This paper
proceeds with numerical convergence tests as if the problems are not
stiff, and hence will add confusion to the question of order of
accuracy.}
\begin{itemize}
  \item We have made it clearer that throughout the manuscript that
  order reduction due to stiffness is observed in our formulation and
  examples.
  \item While we would like to reduce the order reduction by
  interpreting SDC as a preconditioner of the Gaussian collocation
  discretization [21], this is currently not practical with the
  complexity of our governing equations.  The number of unknowns per
  time step is $3N\times(\text{number of vesicles})  +
  2N_{\mathrm{wall}} \times (\text{number of walls})$ (3 = 2 (for
  position) + 1 (for tension)).  For the final example (Couette), this
  corresponds to 6656 unknowns which discretize the non-linear
  integro-differential equations with an algebraic constraint.  We are
  unaware of a Krylov-Deferred-Correction method being applied to
  systems of this size and complexity.
  \item These estimates become even more prohibitive in 3D which is an
  extension that we are currently considering.
  \item Since we are often not in the asymptotic regime when doing
  adaptive time stepping, we are using the safety factors defined by
  $\alpha$, $\beta_{\mathrm{up}}$, and $\beta_{\mathrm{down}}$.  We have
  found, and demonstrated in the manuscript, that we are able to achieve
  user-defined tolerances, even when SDC iterations are naively applied
  to the stiff equations.  We now mention that these parameters are
  needed because of observed order reduction.
\end{itemize}

\comment{Finally, the formulas like that for $C_{A}$ appearing on page
11, directly use assume the order of the scheme. In the stiff limit,
these are only valid when the SDC iterations have converged. So it would
seem that a to use this sort of estimate for higher-order SDC, one would
need to monitor the residual as well.}
\begin{itemize}
  \item We experimented with monitoring the residual of the system as
  well.  The challenge is that the governing equations are sufficiently
  ill-conditioned, and we found that a small residual still resulted in
  large errors in the area and length.  However, we do observe that the
  residual converges to zero with respect to the SDC iteration.  This is
  now mentioned in the text, but numbers are not reported.
  \item Even though the expression involving $C_{A}$ and other similar
  expressions rely on the order of the method being achieved, without
  further calculations, this is the best estimate we have available.
  Experimentally, these assumptions are justified since we do achieve
  the desired tolerance in many examples.  Moreover, the safety factors
  are used to help increase the likelihood of a time step being accepted
  even if the method is not in the asymptotic regime.
\end{itemize}


\comment{In my opinion, the use of only the length and area of the
vesicles as a measure of error is a major short-coming in the numerical
experiments, particularly in light of the results on the demonstrated
order of accuracy. The SDC method is applied to the position of vesicle
markers, hence the most appropriate measure of error would be a norm of
the error in position at a fixed time. This of course has the
disadvantage that the error would have to be approximated by comparison
to a reference solution, but this is not reasonable grounds for
exclusion of such results. Accurate computation of particle trajectories
will imply that the vesicle length and area remain accurate, but the
converse is not true. For simulations with many vesicles moving in
complex flows, the exact position of each may not be the most relevant
finding, but in terms of demonstrating the accuracy and order of
convergence of a new numerical approach, convergence of position error
should be demonstrated.}
\begin{itemize}
  \item We agree with the reviewer that convergence of the error in area
  and length is not sufficient to guarantee convergence of the vesicle
  position.  In our previous paper, we did a convergence study of the
  vesicle-wall position of the stenosis example (Table 9). In all of our
  experiments when the error in area and length are small, the error in
  the position is small. This is an empirical observation that we use
  here. 

  \item In this manuscript, we now also do a convergence study of the
  vesicle position for the relaxation flow.  We computed a
  highly-accurate (with a very small fixed-time stepping) reference
  solution to use as an ``exact" solution.  As desired, the errors in
  area and length behave asymptotically in the same manner as the error
  in position.  For the fixed time step size, this is reported as an
  additional column in Tables 2-4.  For the adaptive time stepping, we
  added additional plots (Figure 2) that show how the error in length
  and area achieve the desired tolerance, and that the error in
  position decays at the same rate as the error in length and position. 

  \item Now that we have demonstrated convergence for this example, we
  feel confident that using the error in area and length is suitable
  for doing adaptive time stepping.  This is important since there is
  now a negligible amount of additional cost to estimate the local
  truncation error.
\end{itemize}

\comment{A related issue: How are the length and area of the vesicles
actually computed? Is it the case that these are spectrally accurate as
well?}
\begin{itemize}
  \item The vesicle boundary is represented in a Fourier basis.
  Therefore the area and length can be computed with spectral accuracy
  in space.  This is now mentioned in the text at two locations
  (Sections 4 and 5).
\end{itemize}

\comment{pg 1. The first sentence of the abstract mentions an
``arbitrary-order accurate" method. Given the ambiguous evidence of even
3rd order accuracy in time, I think this characterization is
inappropriate.}
\begin{itemize}
  \item We changed it to ``high-order accurate".  We added a sentence to
  the abstract stating that we achieve in excess of second-order
  accuracy, which we do observe.
  
  \item We added an additional row to the second numerical example to
  help show that the error is decaying close to the expected rate, but
  that additional sources of error are dominating the overall error.
\end{itemize}

\comment{pg.~2 first paragraph. ,”implicit higher-order methods are
quite challenging to combine with adaptive schemes. The most common
methods are implicit multistep schemes, but those are problematic
especially for dynamics that involve moving interfaces.” I am not sure
here what the authors mean. Combining multistep methods with adaptive
time stepping does require more work than single step methods, but that
holds true for both implicit and explicit. Why do moving interfaces make
the things ``especially problematic"?}
\begin{itemize}
  \item We wanted to mention that adaptive time stepping with multistep
  methods is problematic if the time step size changes too quickly.  The
  method can become unstable which we now state and reference a paper.
\end{itemize}

\comment{pg 2 second paragraph. You might explain why a standard stiff DIRK
method does not meet you design goals.}
\begin{itemize}
  \item The problem with a DIRK method is that we are unaware of a
  method for integro-differential equations such as the ones we have to
  solve.  We have had success in the previous papers with using IMEX
  splittings, both first- and second-order.
  \item The issue with second-order adaptive IMEX is that it is fragile
  if the time step size changes too quickly.  Therefore, we want to only
  use the first-order time stepper and then SDC sweeps to increase the
  accuracy.
\end{itemize}

\comment{pg 2, ``The incompressibility and inextensibility conditions
require that the vesicle preserves both its enclosed area and total
length." You should be clear that this is true only in two dimensions.}
\begin{itemize}
  \item This is clarified
\end{itemize}

\comment{pg 2., ``These are relatively easy to implement but inefficient
due to bending stiffness." It is not the bending stiffness, but rather
the small time step that the bending stiffness necessitates.}
\begin{itemize}
  \item This is corrected.
\end{itemize}

\comment{pg 2, ``grouped in three classes of methods" There are no
citations given for the first two classes. Do they actually exist?}
\begin{itemize}
  \item We have now cited cited works that were in the bibliography
  under the appropriate classes.
\end{itemize}

\comment{pg 2, and beyond: ``a first order backward Euler scheme"
``first order" should be hyphenated when used as an adjective.}
\begin{itemize}
  \item This is fixed and we searched for all other instances of this
  same error.
\end{itemize}

\comment{pg 4. ``Any time stepping scheme can be used to generate a
provisional solution".  Time stepping schemes do not produce a
continuous function of t.  This explanation of SDC mixes up approximate
continuous solutions and approximate discrete solutions.  The former can
be constructed from the latter by standard interpolation.  The notation
needs to cleared up here to differentiate between discrete and
continuous approximate solutions.}
\begin{itemize}
  \item We made it clear that the provisional solution can be formed by
  interpolating the result of a time stepping scheme.
  \item We have updated other parts of the manuscript to help clarify
  where the approximates are continuous and discrete.
\end{itemize}

\comment{ pg. 5, the equation for vesicle suspensions contain a
confusion mix of $\xx_{k}$, $\xx$, $\yy$, $\xx$'.}
\begin{itemize}
  \item We have improved these equations so that they only use $\xx_{k}$
  and $\xx_{j}$.
\end{itemize}

\comment{pg. 7. This section at the top of the page would be much
clearer if it was described on a timestep $[t_{n}, t_{n+1}]$. This is almost
done in the first line of the page, but below the integrals always say 
$t \in [0, T]$, which is misleading at best.}
\begin{itemize}
  \item We have reformulated the presentation of the section at the end
  of Section 2.2.  It is now formulated as the reviewer mentioned.
  \item We used the same presentation in the Appendix.
\end{itemize}

\comment{pg 10 paragraph starting ``A common strategy of an adaptive
time step method is to control an estimate of the local truncation
error. One estimate is the difference of two numerical solutions".  Some
specific citations are appropriate here.}
\begin{itemize}
  \item We have added appropriate citations.
\end{itemize}

\comment{pg 11 ``it is dangerous to change the time step size ..." What
is the danger?}
\begin{itemize}
  \item This has been rephrased.
\end{itemize}

\comment{pg 13 first paragraph in 5.1, the maximum accuracy using 5
Lobatto nodes is 8, not seven.}
\begin{itemize}
  \item We fixed the multiple locations where this error was in the
  manuscript.
\end{itemize}

\comment{Table 1 and beyond: These tables are tedious. Log plots of
errors make understanding convergence so much easier.}
\begin{itemize}
  \item While we agree that log-log plots for a fixed time step size
  would neatly demonstrate the convergence rates, this is not
  the case for the adaptive time stepping results.  We want the readers
  to be able to compare the amount of CPU time required to achieve a
  desired accuracy using a fixed versus an adaptive time step size.

  \item Moreover, for a fixed time step size, we want to report both the
  CPU time, the error in area, and the error in length, and the error in
  position for the relaxation flow.  Combining all these lines for all
  the different time integrators in a log-log plot would also be
  tedious.
\end{itemize}

\comment{Table 5-9. There is too much content in the captions of these
tables. Some is redundant and should be omitted while some should be
moved to the discussion.}
\begin{itemize}
  \item We like to repeat information in the caption that can also be
  found in the text.  This allows readers to quickly understand the
  information that is presented in figures and tables without having to
  search through the text which, at times, is on different pages.
\end{itemize}

\comment{pg 18, first bullet. It makes no sense that 6,000 fixed time
steps gives the same number of digits as 30,000 adaptive time steps. Is
this a typo?}
\begin{itemize}
  \item While this seems counterintuitive, this results are correct.
  The reason is that with a fixed time step size, almost no error is
  committed for the entire simulation, except when the vesicle passes
  through the constriction.  However, with an adaptive time step size, a
  much larger time step is taken when the vesicle is not in the
  constriction, but an extremely small time step size is required when
  the vesicle passes through the constriction.

  \item The net result is that first-order adaptive time stepping
  requires more time steps than first-order fixed time stepping.  This,
  however, is not the case for second-order which further justifies our
  efforts to form second- and higher-order results.

  \item We have clarified this in the text.
\end{itemize}

\comment{Figure 6. Again, the caption is very long.}
\begin{itemize}
  \item Again, we like to repeat information in the caption and the
  text.
\end{itemize}

\comment{pg 23, last bullet. In $3d$, the use of length and area become
volume and surface area. Are these quantities readily computable with
the accuracy necessary to choose the time step?}
\begin{itemize}
  \item Yes they are.  In $3D$, we use a spherical harmonic
  representation which amends itself easily to computing the surface
  area and volume.  This is now discussed in the text.
\end{itemize}

\comment{pg 24-25 the SDC sweeps described here look like procedures for
stepping from $[t_{n}, t_{n+1}]$ where typically SDC is described by
sub-stepping between quadrature nodes in $[t_{n}, t_{n+1}]$ (often
denoted $t_{m}$).  Also, one of the novel things about this method is
the mixing of terms both implicit and explicit and from SDC iteration k
and $k+1$ (or here tilde and not tilde). This should be emphasized in
the description.}
\begin{itemize}
  \item We have emphasized that we are using substepping in Section A.1.
  \item We mentioned that we are using first-order IMEX, which, by
  definition, has a mixture of implicit and explicit terms.  We also
  cite our previous paper which was the first to describe the IMEX
  method that we use.
\end{itemize}


\end{document}
