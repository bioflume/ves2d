Vesicles are deformable capsules filled with a viscous fluid. Vesicle
flows refer to flow of vesicles that are suspended in a Stokesian
fluid.  Vesicles are differentiated from other capsules in the balance
of forces in their interface.  In particular, their boundary is locally
inextensible (in 3D is locally incompressible).  Vesicle flows find
applications in many biological applications such as the simulation of
biomembranes~\cite{sac1996} and red blood
cells~\cite{ghi:rah:bir:mis2011, kao:tah:bib:ezz:ben:bir:mis2011,
mis2006, nog:gom2005,poz1990}.

The dynamics of a vesicle is governed by bending, tension (enforces
inextensibility), hydrodynamic forces from other vesicles in the
suspension, and possibly hydrodynamic forces from flow confinement
boundary walls.  Vesicle suspensions are modeled using the Stokes
equations, a jump in the stress across each vesicle to match the
interfacial forces on the membrane of the vesicle, and a no-slip
boundary condition on the vesicles and the confinement walls.

A significant challenge in simulating vesicle flows is that their
governing equations are stiff. One stiffness source is associated with
the interfacial forces, bending forces, and inextensibility related
tension forces (these depend mainly on the curvature of the vesicle).
Another stiffness source is the hydrodynamic interaction between
vesicles (these depend on the minimum distance between vesicles). A
third stiffness source is the hydrodynamic interaction between vesicles
and external boundary walls (these depend on the minimum distance
between the vesicles and the wall).  Therefore, to resolve the dynamics
of a vesicle, it may be necessary to take a small time step when the
vesicle has regions of large curvature, or when a vesicle approaches a
boundary wall, or approaches another vesicle. However, a large time
step may be taken when a vesicle has a smooth boundary and is separated
from all other vesicles and the boundary walls.  These considerations
necessitate that we use an adaptive time-stepping scheme, mostly for
robustness of the code and to remove the need to manually select a time
step size.

Another challenge in simulating vesicle flows is that maintaining good
accuracy for long horizons requires a great number of time steps when
using a low-order time-stepping method.  One remedy to accumulating
error is to use artificial forces and algorithms that correct each
vesicle's area and length (these quantities are conserved by the
physics)~\cite{bea:rio:seo:bib:mis2004, bib:mis2003, can:kas:mis2003,
ala:ege:low:voi2014}.  However, these approaches may not successfully
stablize the dynamics if too large of a time step is taken.
Alternatively, higher-order methods such as diagonally-implicit
Runge-Kutta (DIRK) or implicit-explicit (IMEX) multistep can mitigate
error accumulation over long simulation times.  While DIRK is
high-order, we are unaware of a formulation for a set of
integro-differential equations, such as the ones that vesicle
suspensions satisfy.  Moreover, when applying many DIRK methods to
stiff algebraic-differential equations, order reduction is often
present~\cite{bos2007}.  While order reduction of DIRK methods can be
reduced, we choose to develop high-order accurate methods with
time-stepping methods that we have successfully used in our previous
work: constant time step size IMEX multistep methods.  Unfortunately,
unless IMEX-Euler is used, these become unstable if the time step size
is changed too quickly~\cite{dah:lin:nev1983}.  Therefore, we will be
developing high-order solutions by iteratively applying an IMEX-Euler
method.

In summary, the design goals for a time-stepping scheme for vesicle
flows is to address stiffness with reasonable computational costs,
allow for adaptivity, and enable high-order time marching (at least
second-order). In the past (in other groups as well as our group),
methodologies have addressed parts of the design goals above, but no
method, to our knowledge, addresses all of them.  In this paper, we
propose a scheme that provides this capability.

%%%%%%%%%%%%%%%%%%%%%%%%%%%%%%%%%%%%%%%%%%%%%%%%%%%%%%%%%%%%%%%%%%%%%%%%
\paragraph{Summary of the method and contributions}
To address stiffness, we use our recent work in which wall-vesicle,
vesicle-vesicle, and bending and tension self-interactions are treated
implicitly~\cite{qua:bir2014b}.  There we used backward difference
formulas (BDFs) with constant time step sizes.  As we discussed, it is
possible to extend BDF with adaptive time-stepping, but, since BDFs
require the solution from multiple previous time step sizes, they
become quite difficult to use in practice~\cite{dah:lin:nev1983}. We
develop a new high-order method that uses spectral deferred correction
(SDC) to iteratively increase the order of a first-order time
integrator.  By introducing SDC, we are able to iteratively construct
high-order solutions that only require the solution from the previous
time step.

In addition, we propose a time-stepping error estimation specific to
vesicle flows.  The incompressibility and inextensibility conditions
impose that two-dimensional vesicles preserve both their enclosed area
and total length. Thus, errors in area enclosed by a vesicle and errors
in its perimeter can be used to estimate the local truncation error.
In this way we avoid forming multiple expensive solutions to estimate
the local truncation error.  We numerically verify that this estimate
is valid by comparing it to the truncation error of the vesicle
position.

Our contributions are summarized below.
\begin{itemize}
  \item We propose an SDC formulation to construct high-order solutions
  of an integro-differential-algebraic equation that governs vesicle
  suspensions.

  \item We propose an adaptive time-stepping method that uses conserved
  quantities to estimate the local truncation error, therefore not
  requiring multiple numerical solutions to form this estimate.

  \item We conduct numerical experiments for several different flows
  involving multiple vesicles, confined flows, and long time horizons
  that demonstrate the behavior of our scheme.
\end{itemize}


%%%%%%%%%%%%%%%%%%%%%%%%%%%%%%%%%%%%%%%%%%%%%%%%%%%%%%%%%%%%%%%%%%%%%%%%
\paragraph{Related work}
There is a rich literature on numerical methods for Stokesian
particulate flows. We only discuss some representative time-stepping
methods for vesicle flows and we omit details on the spatial
discretization. Most methods for vesicle flows are based on integral
equation formulations~\cite{ram:poz1998, soh:tse:li:voi:low2010,
suk:sei2001, zha:sha2013b,
rah:las:vee:cha:mal:moo:sam:shr:vet:vud:zor:bir2010,
rah:vee:bir2010,vee:gue:zor:bir2009, vee:rah:bir:zor2011,zha:sha2011a},
but other formulations based on stencil discretizations also
exist~\cite{laa:sar:mis2014, bib:kas:mis2005, du:zha2008, kim:lai2010}.

The time-stepping schemes used for vesicle flows can be grouped in
three classes of methods. The first class is fully explicit methods,
which often include a penalty formulation for the inextensibility
constraint~\cite{suk:sei2001, soh:tse:li:voi:low2010, ram:poz1998,
kim:lai2010, laa:sar:mis2014, hsi:lai:yan:you2013}.  These are
relatively easy to implement but a small time step is necessitated by
the bending stiffness. Another class of methods treats
self-interactions using linear semi-implicit
time-stepping~\cite{vee:gue:zor:bir2009, rah:vee:bir2010, zha:sha2013b,
zha:sha2011a}.  This addresses one main source of stiffness but they
are fragile. No adaptive or higher than second-order scheme have been
reported for these schemes. Finally, a third class of methods treat all
the vesicle interactions using linearization and semi-implicit
time-stepping. We are aware of two papers in this direction.  In
\cite{zha:sha2013b} a first-order backward Euler scheme is used (with
no adaptivity). As we mentioned in our own work~\cite{qua:bir2014b,
rah:vee:zor:bir2012}, we used a BDF scheme for vesicle flows that
treats all interactions implicitly, but again, it is not adaptive and
is accurate only up to second-order.

Finally, we briefly review the literature on SDC methods.  SDC was
first introduced by Dutt et al.~\cite{dut:gre:rok2000}.  SDC has been
applied to partial differential equations~\cite{bou:lay:min2003,
chr:hay:ong2012, emm:min2012, min:spe:bol:emm:rup2015,
spe:rup:emm:min:bol:kra2014, jia:hua2008, spe:rup:emm:bol:kra2014},
differential algebraic equations~\cite{bu:hua:min2012,
hua:jia:min2007}, and integro-differential
equations~\cite{hua:lai:xia2006}, but not to vesicle flows.
In~\cite{min2003}, Minion was the first to apply SDC to
implicit-explicit (IMEX) methods by studying the problem $\dot{\xx}(t)
= F_{E}(\xx,t) + F_{I}(\xx,t)$ where $F_{E}$ is non-stiff and $F_{I}$
is stiff.  Unfortunately, our governing equations do not exhibit an
additive splitting as non-stiff and stiff terms.  Moreover, IMEX SDC
often often suffers from order reduction, meaning that a very small
time step is required before the asymptotic convergence rate is
achieved.  We will see similar behavior in our work.

Algorithmically, SDC has been accelerated using several strategies.
SDC can be parallelized by either assigning each iteration of SDC to
its own processor~\cite{chr:ong2011, chr:mac:ong2010, chr:hay:ong2012,
ong:mel:chr2012}, or by embedding SDC in parareal or multigrid
framework~\cite{emm:min2012, spe:rup:emm:bol:kra2014,
min:spe:bol:emm:rup2015, spe:rup:emm:min:bol:kra2014}.  We anticipate
that similar parallelization strategies are possible for vesicle
suspensions, but we do not report results at this time.  Next,
high-order (greater than first-order) methods can be used at each SDC
iteration, and this is analyzed in~\cite{chr:ong:qui2009} for the
initial value problem $y' = f(y,t)$.  Finally, by interpreting SDC as
an iterative method that converges to the collocation scheme (a
fully-implicit Runge-Kutta discretization), several methods of
accelerating the convergence have been
investigated~\cite{hua:jia:min2006, hua:jia:min2007, emm:min2012,
wei2013, win:spe:rup2014, bu:hua:min2009, bu:hua:min2012, jia:hua2008,
min:spe:bol:emm:rup2015, spe:rup:emm:bol:kra2014,
spe:rup:emm:min:bol:kra2014, spe:rup:min:emm:kra2014}.  However, as a
first step towards applying SDC to vesicle suspensions, we investigate
the use of a first-order method applied to a predetermined number of
SDC iterations.

%%%%%%%%%%%%%%%%%%%%%%%%%%%%%%%%%%%%%%%%%%%%%%%%%%%%%%%%%%%%%%%%%%%%%%%%
\paragraph{Limitations}
The novelty of this work is to introduce a high-order adaptive time
stepper.  This addresses one of the limitations form our previous
work~\cite{qua:bir2014b}, but also introduces new limitations.
\begin{itemize}
  \item We cannot provide a proof on the convergence order of our
  scheme.  We will see in Section~\ref{s:results} that each SDC
  iteration reduces the error significantly. However, when large
  numbers of SDC iterations are used, the asymptotic convergence rates
  are unclear.  Many other groups have observed similar order
  reduction, and several methods of reducing its effect include Krylov
  deferred corrections~\cite{hua:jia:min2006, hua:jia:min2007,
  bu:hua:min2012, bu:hua:min2009, jia:hua2008}, multilevel
  SDC~\cite{emm:min2012, min:spe:bol:emm:rup2015,
  spe:rup:emm:min:bol:kra2014, spe:rup:emm:bol:kra2014}, and diagonally
  implicit Runge-Kutta sweeps~\cite{wei2013}.  Understanding the
  asymptotic rates of convergence, and resolving the observed order
  reduction, could be used to further optimize the method for choosing
  optimal time step sizes.

%  \item We have not explored the scheme for vesicle suspensions
%  where the viscosity inside and outside each vesicles differs
%  (viscosity contrast).  The difficulty arises because flows with a
%  viscosity contrast include a double-layer potential of the fluid
%  velocity. 

  \item Currently, each vesicle must use the same time step size.
  Therefore, in flows with multiple vesicles, the time step size is
  controlled by the vesicle requiring the smallest time step.  This can
  result in the actual global error being much less than the desired
  error (see {\em Couette} example).  Simulations could be accelerated
  if each vesicle had its own time step size, resulting in a multirate
  time integrator.

\end{itemize}

Let us remark that we do not use adaptivity in space.


%%%%%%%%%%%%%%%%%%%%%%%%%%%%%%%%%%%%%%%%%%%%%%%%%%%%%%%%%%%%%%%%%%%%%%%%
\paragraph{Outline of the paper}
In Section~\ref{s:formulation}, we briefly discuss SDC for initial
value problems (IVPs), and then discuss how to extend SDC to vesicle
suspensions.  In Section~\ref{s:numerics}, we discretize the
differential and integral equations arising in the SDC framework,
discuss preconditioning, and provide complexity estimates.
Section~\ref{s:adaptive} discusses our adaptive time-stepping strategy,
and numerical results are presented in Section~\ref{s:results}.
Finally, we make concluding remarks in Section~\ref{s:conclusions}, and
we reserve a more in depth formulation of our governing equations in
the SDC framework in Appendix~\ref{a:appendix1}.

%%%%%%%%%%%%%%%%%%%%%%%%%%%%%%%%%%%%%%%%%%%%%%%%%%%%%%%%%%%%%%%%%%%%%%%%
\paragraph{Notation}
In Table~\ref{t:notation}, we summarize the main notation used in this
paper.
\begin{table}[htp]
\colorbox{gray!20}{
\begin{tabular}{|l|l|}
\hline
Symbol & Definition \\
\hline
$N$                         & Number of points per vesicle \\
$M$                         & Number of vesicles \\
$N_{t}$                     & Number of time steps \\
$\gamma_{k}$                & Boundary of vesicle $k$ \\
$\xx_{k}(s,t)$              & Parameterization of $\gamma_{k}$ at time
                              $t$ and parameterized in arclength $s$ \\
$\sigma_{k}(s,t)$           & Tension of vesicle $k$ at time $t$ and
                              parameterized in arclength $s$ \\
$\Gamma$                    & Boundary of the confined geometry \\
$\BB(\xx_{k})$              & Bending operator due to vesicle $k$ \\
$\TT(\xx_{k})$              & Tension operator due to vesicle $k$ \\
$\Div(\xx_{k})$             & Surface divergence operator due to 
                              vesicle $k$ \\
$\SS(\xx_{j},\xx_{k})$      & Single-layer potential due to vesicle $k$ 
                              and evaluated on vesicle $j$ \\
$\DD(\xx_{j},\Gamma)$       & Double-layer potential due to $\Gamma$ 
                              and evaluated on vesicle $j$ \\
$\vv_{\infty}(\xx_{j})$     & Background velocity due to a far-field 
                              condition \\
$\vv(\xx_{j};\xx_{k})$      & Velocity of vesicle $j$ due to
                              hydrodynamic forces from vesicle $k$ \\
$\rr(t;\txx{})$             & Residual of equation~\eqref{e:picardEqn} 
                              due to the provisional solution 
                              $\txx{}$ \\
$p$                         & Number of Gauss-Lobatto points used to
                              approximate $\rr$ \\
$\exx{j}$                   & Error between the exact vesicle position and
                              the provisional position $\txx{j}$  \\
$\esigma{j}$                & Error between the exact vesicle tension and
                              the provisional tension $\tsigma{j}$  \\
$A(t),L(t)$                 & Area and length of the vesicles at time 
                              $t$ \\
$e_{A},e_{L}$               & Relative error in area and length \\
$\epsilon$                  & Desired final tolerance for the area and 
                              length of the vesicles \\
$\beta_{\up} \geq 1$        & Maximum rate that the time step is
increased per time step \\
$\beta_{\down} \leq 1$      & Minimum rate that the time step is
decreased per time step \\
$\alpha \leq 1$             & Multiplicative safety factor for the new
                              time step size \\
\hline
\end{tabular}
}
\mcaption{Index of frequently used notation.}{t:notation}
\end{table}

