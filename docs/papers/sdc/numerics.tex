In this section, we present our numerical scheme for
solving~\eqref{e:picardEqn},~\eqref{e:picardRes},
and~\eqref{e:sdcUpdate}.  Evaluating $\vv$ involves layer potentials
and Fourier differentiation, and for concentrated suspensions, this can
be costly, especially when this work is extended to three dimensions.
Moreover, high-order derivatives in the governing equations introduce
stiffness, and this can result in order reduction.  In this first
implementation of SDC for vesicle suspensions, we opt to use the
following strategy: we fix the number of SDC iterations and apply our
previous first-order semi-implicit time integrator~\cite{qua:bir2014b}
to~\eqref{e:picardEqn} and \eqref{e:sdcUpdate}, and use standard
quadrature to compute the residual defined in~\eqref{e:picardRes}.  In
addition, there are several approximations when comparing our
discretization of~\eqref{e:sdcUpdate} with standard methods of
discretizing~\eqref{e:picardCorrect}.  More expensive discretizations
are possible, but we found that they did not improve the overall
accuracy of the method.  We also discuss how the block-diagonal
preconditioner outlined in~\cite{qua:bir2014b} is applied.  We conclude
with estimates of the overall work per time step as a function of the
number of vesicles and the number of points per vesicle.

%%%%%%%%%%%%%%%%%%%%%%%%%%%%%%%%%%%%%%%%%%%%%%%%%%%%%%%%%%%%%%%%%%%%%%%%
\subsection{Spatial Discretization}

Following our previous
work~\cite{qua:bir2014b,rah:vee:bir2010,vee:gue:zor:bir2009}, we use a
Lagrangian formulation by marking each vesicle with $N$ tracker
points.  Since vesicles are modeled as smooth closed curves, all the
derivatives are computed spectrally using the fast Fourier transform
(FFT).  Near-singular integrals are handled using the near-singular
integration scheme outlined in~\cite{qua:bir2014b}.  Finally, we use
Alpert's high-order Gauss-trapezoid quadrature rule~\cite{alp1999} with
accuracy $\bigO(h^{8} \log h)$ to evaluate the single-layer potential,
and the trapezoid rule for the double-layer potential which has
spectral accuracy since its kernel is smooth.

%%%%%%%%%%%%%%%%%%%%%%%%%%%%%%%%%%%%%%%%%%%%%%%%%%%%%%%%%%%%%%%%%%%%%%%%
\subsection{Temporal Discretization}

A fully explicit discretization of~\eqref{e:diffEqn} results in a stiff
system for multiple reasons.  A stiffness analysis
in~\cite{vee:gue:zor:bir2009} reveals that the leading sources of
stiffness, and the corresponding time step restrictions, are:
\begin{itemize}
  \item $\SS(\xx_{j},\xx_{j})\BB(\xx_{j})(\xx_{j})$ (self-hydrodynamic
  bending force) -- $\Delta t \sim \Delta s^{3}$;

  \item $\SS(\xx_{j},\xx_{j})\TT(\xx_{j})(\xx_{j})$ (self-hydrodynamic
  tension force) -- $\Delta t \sim \Delta s$;

  \item $\Div(\xx_{j})(\xx_{j})$ (self-inextensibility force) -- $\Delta
  t \sim \Delta s$;

  \item $\SS(\xx_{j},\xx_{k})(\BB(\xx_{k})(\xx_{k}) +
  \TT(\xx_{k})(\sigma_{k}))$, $j \neq k$ (inter-vesicle hydrodynamic
  forces) -- depends on the inter-vesicle distance.
\end{itemize}
The leading sources of stiffness result from the intra-vesicle
interactions; but, for concentrated suspensions, inter-vesicle
interactions become significant and introduce stiffness.  In future
work, we plan to use spectral analysis to analyze the stiffness due to
inter-vesicle interactions.  In particular, we hope to find a
relationship between the stable time step size and the distance between
vesicles.

To address these multiple sources of stiffness, we use a variant of
IMEX~\cite{asc:ruu:wet1995} time integrators.  IMEX methods were
developed for the problem $\dot{\xx}(t) = F_{E}(\xx) + F_{I}(\xx)$,
where $F_{E}$ is non-stiff and is treated explicitly, and $F_{I}$ is
stiff and is treated implicitly.  For problems with this additive
splitting, the family of time integrators is
\begin{align*}
  \frac{\beta \xx^{m+1} - \xx^{0}}{\Delta t} = 
    F_{E}(\xx^{e}) + F_{I}(\xx^{m+1}),
\end{align*}
where $\xx^{0}$ and $\xx^{e}$ are linear combinations of previous time
steps and $\beta > 0$.  Unfortunately,~\eqref{e:diffEqn} does not have
an additive splitting between stiff and non-stiff terms.  However, we
have observed first- and second-order convergence~\cite{qua:bir2014b}
for the time integrator
\begin{align*}
  \frac{\beta \xx^{m+1}_{j} - \xx^{0}_{j}}{\Delta t} =
    \sum_{k=1}^{M} \vv(\xx_{j}^{e};\xx_{k}^{m+1}), \quad j=1,\dots,M,
\end{align*}
where
\begin{align*}
  \vv(\xx_{j}^{e};\xx_{k}^{m+1}) = \SS(\xx^{e}_{j},\xx^{e}_{k})
    (-\BB(\xx^{e}_{k})(\xx^{m+1}_{k}) +
      \TT(\xx^{e}_{k})(\sigma^{m+1}_{k})).
\end{align*}
Note that in this formulation, it is the operators involved in $\vv$,
such as the bending $\BB(\xx_{k})$ and the single-layer potential
$\SS(\xx_{j},\xx_{k})$, that are discretized explicitly at $\xx^{e}$.
In line with our previous work, we use the first-order method given by
$\beta = 1$, $\xx^{0}=\xx^{m}$, and $\xx^{e}=\xx^{m}$, and the
second-order backward difference formula (BDF) given by $\beta = 3/2$,
$\xx^{0} = 2\xx^{m} - 1/2\xx^{m-1}$, and $\xx^{e} = 2\xx^{m} -
\xx^{m-1}$.  

%%%%%%%%%%%%%%%%%%%%%%%%%%%%%%%%%%%%%%%%%%%%%%%%%%%%%%%%%%%%%%%%%%%%%%%%
\subsection{Quadrature Formula}
\label{s:vesicleQuadrature}

SDC requires a quadrature formula to approximate~\eqref{e:picardRes},
the residual of the Picard integral.  The quadrature rule affects the
accuracy of SDC (Proposition~\ref{pro:convergence}).  However, the
stability of SDC is also effected by the quadrature rule, and this has
been investigated in depth in~\cite{bou:lay:min2003}.  In order to
achieve the highest possible accuracy, we could use Gaussian nodes;
however, in order to avoid extrapolation, we would like both endpoints
to be quadrature nodes.  Therefore, we use Gauss-Lobatto points, $t_{m}
= t_{1} < \cdots < t_{p} = t_{m+1}$, since they include both endpoints
and have successfully been used by other groups~\cite{bou:lay:min2003,
bou:min2010, min2003}, and have an order of accuracy of $2p-2$.
Alternatively, we could use Radau quadrature formula which include the
right endpoint, but not the left endpoint.

For computational efficiency, the accuracy of the quadrature formula
should not exceed the expected rate of convergence.  This will become
especially important in three dimensions since multiple variables must
be stored at each of the quadrature nodes for future SDC iterations.
In our numerical examples, we take $p$ to be the smallest value such
that the quadrature order of accuracy (in our case, $2p-2$) is greater
than or equal to $n_{\sdc}+1$ (the expected order of convergence of the
time integrator).  Once a quadrature formula for~\eqref{e:picardEqn} is
chosen, the result is a fully-implicit collocation scheme whose
solution the SDC iteration attempts to converge towards.

%%%%%%%%%%%%%%%%%%%%%%%%%%%%%%%%%%%%%%%%%%%%%%%%%%%%%%%%%%%%%%%%%%%%%%%%
\subsection{Picard Integral Discretization}
\label{s:vesiclePicard}

Equations~\eqref{e:picardEqn} and~\eqref{e:sdcUpdate} have similar
structure, and we take advantage of this structure in their
discretizations.  For adaptivity, we want a scheme that easily allows
for variable time step sizes.  For this reason, we use only first-order
methods for~\eqref{e:picardEqn} and~\eqref{e:sdcUpdate}.  When desired,
we use SDC iterations to increase the accuracy.

The first-order provisional solution is found at the substeps
$t_{1},\ldots,t_{p}$ by discretizing~\eqref{e:picardEqn} as
\begin{align}
  \xx_{j}^{m+1} = \xx_{j}^{m} + \Delta t_{m}
    \sum_{k=1}^{M} \vv(\xx_{j}^{m};\xx_{k}^{m+1}), \quad
    m=1,\ldots,p-1,
  \label{e:numericProvisionalImplicit}
\end{align}
where $\Delta t_{m} = t_{m+1} - t_{m}$.  This is exactly the first-order
time integrator we introduced in~\cite{qua:bir2014b}.  As we march in
time with~\eqref{e:numericProvisionalImplicit}, we save the variables
required for near-singular integration for future SDC iterations.  Then,
we evaluate the residual~\eqref{e:picardRes} using the Gauss-Lobatto
quadrature rule.

In line with~\cite{min2003}, which considers semi-implicit SDC methods,
we would like to discretize~\eqref{e:sdcUpdate} as
\begin{align*}
  \ee^{m+1}_{\xx_{j}} = \ee^{m}_{\xx_{j}} + 
  \rr_{j}^{m+1} - \rr_{j}^{m} + 
  \Delta t_{m}\sum_{k=1}^{M}(
    \vv(\txx{j}^{m}+\ee^{m}_{\xx_{j}};\txx{k}^{m+1}+\ee^{m+1}_{\xx_{k}}) - 
    \vv(\txx{j}^{m},\txx{k}^{m+1})).
\end{align*}
The issue with this formulation is that it requires additional storage
and computations to find the velocity due to the vesicle parameterized
by $\txx{j}^{m} + \ee^{m}_{\xx_{j}}$.  Again, in three dimensions
this restriction is even more prohibitive.  We have experimented with
other discretizations of~\eqref{e:sdcUpdate}.  The simplest such one is
\begin{align*}
  \ee^{m+1}_{\xx_{j}} = \ee^{m}_{\xx_{j}} + 
    \rr_{j}^{m+1} - \rr_{j}^{m}.
\end{align*}
This discretization is consistent with the governing
equations\footnote{If $\ee^{m}_{\xx_{j}}$ converges to 0, then
$\rr_{j}^{m+1}=\rr_{j}^{m}$, and by~\eqref{e:picardRes}, SDC has
converged to the solution of the fully-implicit collocation
discretization of~\eqref{e:picardEqn}.}, but, experimentally, SDC
converges only if a very small time step is used.  To allow for larger
time steps, we can include the implicit term in the discretization
\begin{align}
  \label{e:numericSDCimex}
  \ee^{m+1}_{\xx_{j}} = \ee^{m}_{\xx_{j}} + 
  \rr_{j}^{m+1} - \rr_{j}^{m} + 
  \Delta t_{m} \sum_{k=1}^{M} 
  (\vv(\txx{j}^{m};\txx{k}^{m+1}+\ee_{\xx_{k}}^{m+1}) -
    \vv(\txx{j}^{m};\txx{k}^{m+1})),
\end{align}
where we use a slight abuse of notation by defining
\begin{align*}
  \vv(\txx{j}^{m};\txx{k}^{m+1} + \ee_{\xx_{k}}^{m+1}) = 
    \SS(\txx{j}^{m},\txx{k}^{m})
    (-\BB(\txx{k}^{m})(\txx{k}^{m+1}+\ee_{\xx_{k}}^{m+1}) +
      \TT(\txx{k}^{m})(\sigma^{m+1}_{k} + e_{\sigma_{k}}^{m+1})).
\end{align*}
(Note how none of the operators depend on the error $\ee$.)
Appendix~\ref{a:appendix1} presents this same discretization without
the abuse of notation.  While this discretization allows larger time
steps, it does not converge to the solution of the fully-implicit
discretization of~\eqref{e:picardEqn} and is incompatible with the
inextensibility constraint (see Appendix~\ref{a:appendix1}).  As an
alternative to~\eqref{e:numericSDCimex}, an appropriate discretization
of the error equation is
\begin{align}
  \label{e:numericSDCImplicit}
  \ee^{m+1}_{\xx_{j}} = \ee^{m}_{\xx_{j}} + 
  \rr_{j}^{m+1} - \rr_{j}^{m} + 
  \Delta t_{m} \sum_{k=1}^{M} 
  (\vv(\txx{j}^{m+1};\txx{k}^{m+1}+\ee_{\xx_{k}}^{m+1}) -
    \vv(\txx{j}^{m+1};\txx{k}^{m+1})),
\end{align}
where we are using the same abuse of notation.  Note
that~\eqref{e:numericSDCImplicit} only requires evaluating the velocity
field due to the vesicle configuration given by $\txx{j}^{m+1}$.  Since
these velocity fields are required to form residual $\rr$, no
additional velocity fields need to be formed.

To summarize, the main steps for using SDC to solve~\eqref{e:picardEqn}
are
\begin{enumerate}
  \item Find a first-order provisional solution $\tilde{\xx}$ at the
  Gauss-Lobatto quadrature points
  using~\eqref{e:numericProvisionalImplicit}.

  \item \label{step:residual} Compute the residual $\rr$ by
  approximating the integral in~\eqref{e:picardRes} with the
  Gauss-Lobatto quadrature rule.

  \item Use~\eqref{e:numericSDCImplicit} to approximate the error
  $\tee{}$.

  \item Define the new provisional solution to be $\txx{} + \tee{}$.

  \item \label{step:repeat} Repeat
  steps~\ref{step:residual}--\ref{step:repeat} $n_{\sdc}$ times.
\end{enumerate}



%%%%%%%%%%%%%%%%%%%%%%%%%%%%%%%%%%%%%%%%%%%%%%%%%%%%%%%%%%%%%%%%%%%%%%%%
\subsection{Preconditioning}
\label{s:preco}
Equations~\eqref{e:numericProvisionalImplicit}
and~\eqref{e:numericSDCImplicit} are ill-conditioned and require a
large number of GMRES iterations~\cite{vee:gue:zor:bir2009}.  To reduce
the cost of the linear solves, we used a block-diagonal preconditioner
in~\cite{qua:bir2014b} which is formed and factorized in matrix form at
each time step.  Using this preconditioner, the number of
preconditioned GMRES iterations depends only on the magnitude of the
inter-vesicle interactions, which in turn is a function of the
proximity of the vesicles.  For further savings, we freeze and
factorize the preconditioner at the first Gauss-Lobatto point, and this
preconditioner is used for all the subsequent Gauss-Lobatto points and
SDC iterates.  By freezing the preconditioner, there is a slight
increase in the number of GMRES iterations as the solution is formed at
the each subsequent Gauss-Lobbato substep.  However, the savings is
significant when compared to constructing and factorizing the
preconditioner at each substep.  Moreover, we only require one matrix
factorization per time step, which is the number of factorizations
required in the preconditioner we introduced in~\cite{qua:bir2014b}.



%%%%%%%%%%%%%%%%%%%%%%%%%%%%%%%%%%%%%%%%%%%%%%%%%%%%%%%%%%%%%%%%%%%%%%%%
\subsection{Complexity Estimates}
Here we summarize the cost of the most expensive algorithms required in
our formulation.
\begin{itemize}

\item \emph{Matrix-vector multiplication:}
For unbounded flows, if $M$ vesicles are each discretized with $N$
points, the bending and tension calculations require $\bigO(MN\log N)$
operations using the FFT, and the single-layer potential requires
$\bigO(MN)$ operations using the fast multipole method (FMM).  If the
solid wall is discretized with $N_{\wall}$ points, using the FMM, the
matrix-vector multiplication requires $\bigO(MN\log N + N_{\wall})$
operations.

\item \emph{Computing the residual:}
Given a provisional solution $\tilde{\xx}$, its velocity $\vv$
satisfies~\eqref{e:diffEqn} and can be
formed with a single matrix-vector multiplication.  Therefore,
computing the residual $\rr$ requires $p$ matrix-vector
multiplications, one for each Gauss-Lobatto substep, and computing the
residual requires $\bigO(p(MN\log N + N_{\wall}))$ operations.

\item \emph{Forming the provisional solution and SDC iterations:}
Equations~\eqref{e:numericProvisionalImplicit}
and~\eqref{e:numericSDCImplicit} require solving the same linear system
(only the right-hand sides are different), and our preconditioner
results in a mesh-independent number of GMRES iterations.  Therefore,
if $n_{\gmres}$ total iterations are required to find the provisional
solution, then $n_{\sdc}$ SDC iterations requires
$\bigO(n_{\gmres}p(n_{\sdc}+1)(MN\log N + N_{\wall}))$ operations.  

\item \emph{Forming the preconditioner:} The preconditioner is computed
and stored in matrix form and requires $\bigO(MN^{2}\log N)$ operations
per time step by using Fourier differentiation and the FFT.  The
preconditioner must be factorized which requires $\bigO(MN^{3})$
operations.  This is computed only once per time step and is reused at
all the additional Gauss-Lobatto quadrature points.  In two dimensions,
this cost is acceptable since the number of unknowns on each vesicle is
sufficiently small.  However, in three dimensions, this cost is
unacceptable and different preconditioners will need to be constructed.
\end{itemize}


